

% Biochemical Targets
\section{Biochemical Targets}


% fungal infection
\subsection{Fungal Infection}

Evidence has been put forth that fungal infection is detectable
in brain samples from Alzheimer's disease patients.
\cite{alonso2013fungal}
Recently, fungal infection was directly visualized in the brains
of 11 of 11 AD patients and 0 of 10 controls.
\cite{pisa2015different}
Misfolded amyloid proteins may originate from a transmissible infectious process
(like prion disease).
Some proteins behave as prions in yeast and other fungi.
\cite{soto2006amyloids}
Neuroinflammation is associated with innate immunity,
intended to fight infection at the blood-brain barrier (BBB).
\cite{hauwel2005innate}
Fungal infection could explain the connection between
neuroinflammation, innate immunity and the toxic cell
debris associated with AD.
Amyloid-$\beta$ is toxic to yeast cells
as well as nerve cells.
A yeast model of amyloid toxicity has been proposed
because of yeast's susceptibility to the protein.
\cite{treusch2011functional}
Conversely, amyloids are also believed to be
expressed on fungal cell surfaces.
\cite{gebbink2005amyloids}
The relationship between amyloid, prion and fungi is complicated.
\cite{tessier2009unraveling}






%
An early screening of 16 herbs for antifungal activity
found that cloves, cinnamon, mustard, allspice, garlic, and oregano
were promising.
A possible synergism between Potassium sorbate and cloves
was identified.
\cite{azzouz1982comparative}
The main constituent of clove oil is eugenol which
is also present in nutmeg, cinnamon, basil and bay leaf.
\cite{?}
Another screening of 52 herbs for their antifungal activity
against phytopathogenic fungi showed that two herbs from the Apiaceae family,
cumin and black zira, followed by cardamom from the Zingiberaceae family
were active against \textit{Fusarium}, \textit{Verticillium},
\textit{Botrytis} and \textit{Alternaria}.
In our matrix of Asian mixtures we have at least 3 Apiaceae,
Ligusticum,
Angelica,
and Bupleurum.
Ginger represents the Zingiberaceae.


Many screenings of antifungal activity focus on phytopathogenic fungi.
However, all of the species that are suspected to play a role in
AD are known human pathogens, not phytopathogens,
\textit{Candida}
\textit{Cladosporium}
\textit{Malasezzia}
\textit{Neosartorya}
\textit{Phoma}
\textit{Sacharomyces}
and \textit{Sclerotinia}.
Because different fungi are susceptible to different antifungal agents,
and different fungi are resistant to those same agents,
we must narrow our search for antifungal evidence to these species.











\subsection{Innate Immunity}
The innate immune system, preserved from ancient genes,
is involved in response to pathogenic fungi.
\cite{means2009evolutionarily}
The innate immune system in organisms as diverse as humans and plants
is remarkably similar.
\cite{nurnberger2002innate, nurnberger2004innate}
The signalling cascades involved in innate immune functions
are providing insights into infectious disease, autoimmunity and allergy.
The innate immune response to many infectious agents is associated with
inflammation.
\cite{akira2006pathogen}
Inflammation in AD may be mediated by innate immunity,
according to transgenic animal model.
\cite{fassbender2004lps}

Innate immune response to fungal infection is
contributed to by Toll-like receptors (TLRs).
\cite{bellocchio2004contribution}
Immune response to to some pathogenic fungi is dependent
on TLRs.
\cite{
viriyakosol2005innate,
roeder2004toll}


Pathogen associated molecular patterns (PAMPs) are
associated with both fungal infection
\cite{kumar2011pathogen}
and AD.
\cite{salminen2009inflammation}




%Resveratrol is a promising neuroprotective substance.
%\cite{li2012resveratrol}


\subsection{Amyloid-$\beta$}


An animal model found ZMT effective in reducing Amyloid-$\beta$
toxictiy.
\cite{tohda2003repair}


\subsection{Inflammation}


Long-term use of Non-Steroidal Antiinflammatory Drugs (NSAIDs)
have been associated with a dramatic decrease in the risk of AD
in a large cohort.
The risk ratio was 0.20 (95 percent confidence interval, 0.05 to 0.83).
\cite{in2001nonsteroidal}

Other epidemiological evidence suggests that individuals
with diets right in antiinflammatory Curmcumin
are at lower risk of developing AD.
\cite{?}


Inflammation may be the key to initiating the toxic cascade that
characterized Alzheimer's rather than the result of
midfolded proteins such as amyloid and tau.
\cite{heneka2007inflammatory}


Because greater neuroinflammation is observed in human disease
than current animal models,
new animal models are sought with increased neuroinflammation.
One such model is involving increased interleukin-6.
\cite{millington2014chronic}




\subsection{Cholinergic Drugs}

Cholinergic drugs were able to slow the progression of AD
after 2 years of follow-up.
Although, cognitive deterioration was not stopped.
\cite{requena2006effects}




\subsection{Hormonal Therapy}
Postmenopausal hormone therapy was not found
to affect AD risk in a systematic review of
epidemiological studies.
\cite{o2014postmenopausal}


