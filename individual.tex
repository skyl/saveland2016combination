%
% Studies on individual constituents
%

% Individual Natural Products
\section{Evidence for Individual Natural Products}

%In addition to the ## genera identified in
%the mixtures, several additional which are not used
%in the Asian combinations
%but which have promising preclinical and clinical results
%concerning AD
%have been identified.

Natural products are evaluated for their effects
as refined molecules and
as crude drugs.
Molecules of interest in AD include,
polyphenols,
flavonoids,
alkaloids,
phenylpropanoids,
triterpenoid saponins,
and polysaccharides
which are investigated
against multiple pathological pathways.
\cite{
darvesh2010oxidative,
gao2013research}

In addition to the traditional formulations,
individual constituents are studied for their potential
to treat AD including
Huperzine,
Gingko,
Curcuma,
Salidroside,
Periwinkle-vinca,
Centella,
Melissa,
Polygala,
Salvia,
and Withania.
% ......
\cite{sun2013traditional}
Investigation in the West has been limited to a few herbs,
notably Gingko. Some herbs used in traditional European medicine
such as \textit{Salvia officinalis} and \textit{Melissa officinalis}
are under investigation for their potential as AD treatments.
\cite{perry1999medicinal}


% Acorus
\subsection{Acorus}

$\beta$-asarone, a constituent of some
\textit{Acorus} products,
is regulated in the United States and Europe
for its profound adverse effects
at high doses in test animals.
\cite{EC2002directorate}

% General
\textit{Acorus calamus} (Ac),
found in N out of M of the reviewed Asian combinations,
has a long history of medicinal and recreational use
including use in India, China, Thailand and by Native Americans.
\cite{phongpaichit2005antimicrobial}
Ac is known as one of the most important medicines in
the Ayurvedic system.
\cite{kumar2013medicinal}
\textit{Acorus gramineus} is a smaller Japanese plant
that is also used in medicine.

% Psychiatric
A study of \textit{Acorus gramineus}
essential oil as an olfactory stimulant revealed that it was
able to enhance learning and memory in an animal model of AD.
\cite{liu2010study}
The \textit{in vivo} activities of the
\textit{Acorus calamus} root on learning and memory
has long been suspected and studied.
An animal model of learning and memory disability
showed that \textit{Acorus} may be promising drug.
\cite{wenling1993facilitatory}
Essential oil from \textit{Acorus calamus}
displayed \textit{in vitro}
acetylcholinesterase inhibitory activity,
supporting its use for neurological disorders
and dementia in traditional ayurvedic medicine
and in modern practice.
\cite{mukherjee2007vitro}
Ac essential oil had positive effects
in an animal model of depression.
\cite{han2013antidepressant}
Active components of Ac improved learning and memory in
an animal model of dementia.
\cite{guo2012effects}
$\beta$-asarone and eugenol were able to
protect amyloid-$\beta$-injured nerve cells \textit{in vitro}.
\cite{jiang2006protective}
Ac, as part of SZL,
induced no significant adverse effects
and was tolerable by more than 92\% of the participants
in a RCT of 98 AD patients.
\cite{pan2014shen}

% antifungal
Many studies support the notion that Ac products exhibit clinical-relevant
antifungal activity.
A survey of Thai plants found that Ac
exhibited strong antimicrobial activity against spoilage yeasts
and moderate, antioxidant and AChE-inhibiting activities.
% more results for other plants to look at!
\cite{nanasombat2014antimicrobial}
An \textit{in vitro} study showed that $\beta$-asarone,
an active constituent of Ac,
was highly potent against \textit{Candida albicans},
disrupting ergosterol synthesis,
and suggested that $\beta$-asarone could be used as a topical
antifungal
\cite{rajput2013beta}
or as an antifungal in agricultural application.
\cite{lee2004antifungal}

It has been reported that some variations of \textit{Acorus calamus},
namely the American dipoloid varieties,
do not contain $\beta$-asarone.
\cite{phongpaichit2005antimicrobial}
We have not seen a study indicating whether crude extracts
of these non-$\beta$-asarone producing varieties are antifungal.
Other related substances have been found to be potent antifungals.
\cite{rajput2013anti}

Further \textit{in vitro} evidence suggest that
an active fraction of Ac,
by inhibiting ergosterol synthesis,
could help us overcome strains which have become
resistant to current antifungal drugs.
\cite{
subha2008effect,
subha2009vitro}
Researchers have become fairly confident based on \textit{in vitro}
results that $\alpha$ and $\beta$ asarones are responsible for
the pronounced antifungal activity.
\cite{devi2009antimicrobial}
Against \textit{Candida} biofilms,
compared to Amphotericin B and ketoconazole, the gold-standard antifungals,
\textit{Acorus calamus} fractions were superior.
The fungal biomass was completely killed at 2mg/ml concentrations
of the Ac concetrates; but, the standard antifungal drugs
were unable to perfuse the biofilm and had little effect.
\cite{subha2009candida}
Although, an earlier review with a different assay
showed that, while Ac and clove were moderately effective fungistatics
($Ac < clove < eugenol < Amphotericin B$),
Amphotericin B was a far stronger fungicidal agent
compared to \textit{calamus}.
\cite{thirach2003antifungal}
However, recently,
it was shown that asarones from Ac interacted
synergistically with standard antifungals,
requiring less of each to express fungicidal activity.
\cite{kumar2015asarones}
The \textit{in vitro} antifungal effects of Ac
were confirmed \textit{in vivo}.
\textit{Acorus calamus} extract
was superior to ketoconazole in an animal model.
\cite{subha2009combating}





%Perhaps irrelevantly, an endophytic fungus associated with Ac,
%\textit{Fusarium oxysporum},
%had modest antimicrobial and antifungal
%activity.
%\cite{barik2010phylogenetic}



\subsection{Berberine}


Berberine has been demonstrated to be clinically effective
in alleviating type 2 diabetes.
It is hypothesized that the observed positive effects
on obesity and insulin resistance could be related
to alterations in the gut microbiome induced by oral adminstration.
Berbering alleviated inflammation by reducing the exogenous antigen load
produced by the microbiota of the gut.
\cite{zhang2012structural}


Berberine ameliorated $\beta$-amyloid pathology
gliosis, and cognitive impairment in an
animal model of Alzheimer's disease.
A "profound reduction" in $\beta$-amyloid was observed.
\cite{durairajan2012berberine}

Berberine reduced AB levels in human neuroglioma cells.
\cite{asai2007berberine}




% misc
Berberine prevented metastasis and tumor growth in
an animal model of breast cancer.
\cite{refaat2013berberine}

Berberine activates AMPK and thus prevents
tumor metastasis \textit{in vivo}.
\cite{kim2012berberine}
Interestingly, AMPK activation ameliorated AD-like pathology
in an animal model of AD.
\cite{du2015ampk}

Berberine inhibits colon cell proliferation.
This inhibition may be explained by the observation that
berberine down-regulates epidermal growth factor (EGF).
\cite{wang2013berberine}
Surprisingly, EGF receptor activation has been
linke to AD, as well.
\cite{birecree1988immunoreactive}

Berberine was neproprotective in an animal model of
hypertension furthering the idea that berberine
may have diverse beneficial effects in
pathologies related to metabolic syndrome.
\cite{kishimoto2015effects}
Metabolic syndrome is strongly correlated with AD.
\cite{razay2007metabolic}


Berberine acts on the Wnt/$\beta$-catenin pathway
\cite{wu2012berberine}
which may have significance in the treatment of AD.
\cite{esposito2006marijuana}




% glucose
Berberine improves glucose metabolism via multiple pathways,
activating AMPK, inhibiting gluconeogenesis,
and inhibiting lipogenesis in the liver.
\cite{xia2011berberine}
Impaired glucose metabolism is associated with AD.
\cite{baker2011insulin}

A 2012 review of randomized clinical trials of berberine
to treat type 2 diabetes found 14 trials of 1068 patients.
The results suggested that berberine may be able to reduce
blood sugar as well as prescription hypoglycaemics
while improving lipid profile measures.
\cite{dong2012berberine}

In an animal model,
the positive antidiabetic effects of
1 month of berberine-containing TCM treatment
were sustained for 1 year after treatment was ceased.
\cite{zhao2012sustained}


% interaction
A number of cytochrome protien (CYP) enzymes were inhibited
by berberine ingestion in human subjects.
CYP2D6, 2C9, and CYP3A4 activities were decreased.
\cite{guo2012repeated}
This finding indicates that berberine has potential interactions with
other drugs. Berberine interactions could help to improve the
bioavailability of other drugs and create a synergistic
combination. However, care must be taken to insure that
the coadministration with berberine does not result in overdose.
Berberine itself had higher bioavailability when
coadministered with other herbs than when administered alone
or as a crude extract of \textit{Coptidis chinensis},
validating the idea that TCM combinations
have synergistic pharmacokinetics.
% One of the mixtures was "Wen-Pi-Tang-Hab-Wu-Ling-San (WHW),
% a de-coction of 15 herbs .. combining Wen-Pi-Tang and Wu-Ling-San".
% The other was
\cite{chen2013comparative}



% inflammation
Berberine has antiinflammatory properties.
Pro-inflammatory cytokines were decreased in an animal model of colitis.
\cite{yan2012berberine}

Berberine clearly reduced oxidative stress
related to renal ischemia–reperfusion injury
\textit{in vitro}.
\cite{yu2013berberine}
Similarly, berberine reduces expression of cyclooxygenase-2 (COX-2)
and inflammatory prostaglandins.
\cite{singh2011berberine}
COX inhibitors and other NSAIDs are considered neuroprotective
and have potential in AD treatment.
\cite{hoozemans2005role}







\subsection{Curcuma}

Turmeric, \textit{Curcuma longa} root, has a long history of use as medicine.
\cite{?}
Curry consumption in old age may benefit cognitive function
according to an epidemiological study.
\cite{ng2006curry}

A case series reported that turmeric was remarkably effective
in treating 3 patients with AD.
Interestingly, two of these patients were also taking
Yokukansan under the supervision of the Japanese physicians.
\cite{hishikawa2012effects}




% curcumin

Curcumin, a phenolic compound derived from turmeric,
has been investigated for various neurological disorders including
major depression, tardive dyskinesia and diabetic neuropathy.
\cite{kulkarni2010overview}
Curcumin has been effective in animal models of AD.
\cite{?}
The effects are hypothesized to be antioxidant, antiinflammatory,
acetylcholinesterase-inhibiting,
and AB inflammation-inhibiting.
\cite{?}
\textit{In vivo} studies have show that
the bioavailability of curcumin alone is poor.
\cite{?}
However, it has been show that curcumin can pass the BBB.
Also, some compounds may be able to increase the bioavailability
of curcumin. In one study, piperine, an alkaloid from black pepper,
was able to increase curcumin concentrations in the
serum of human volunteers by 2000\%.
\cite{shoba1998log}
An effort has been made make curcumin more soluble in water
through the use of synthetic nanoparticles.
\cite{mathew2012curcumin}

The success of curcuminoids in preclinical models of AD has been impressive,
%salvaging the expression of multiple relevant targets and
suppressing presenilin,
reducing the burden of AB and tau,
modulating the immune system,
increasing AB efflux,
and enhancing spatial memory.
\cite{
yoshida2011turmeric,
shytle2012optimized,
ahmed2010curcuminoids,
ahmed2011comparative,
zhang2006curcuminoids,
villaflores2012effects}

At least 4 clinical trials have been initiated
to test the efficacy of curcumin in treating AD.
No trial has been done on a crude extract of turmeric.(?)

% cruder turmeric

However, preclinical reviewers have emphasized that non-curcumin
constituents of turmeric may be important in the overall therapeutic
action of turmeric.
\cite{ahmed2014therapeutic}
Human safety trials have shown that curcumin is relatively safe
up to 12 grams a day.
However, a review of clinical trials evaluating curcumin
in AD have failed to establish curcumin as an effective drug.
\cite{hamaguchi2010review}

Compounds other than curcumin may be important to
prevent cognitive decline in the elderly
and even to help cure AD.
In addition to curcumin, turmeric contains small molecules
that were able to protect cell cultures from AB.
\cite{park2002discovery}



\subsection{Salvia}

Salvia lavandulaefolia (Spanish sage)
had strong in vitro antiinflammatory activity relevant to AD.
\cite{perry2001vitro}
Salvia is a large genus with antifungal properties.
\cite{yuce2014essential, tabanca2006chemical}



\subsection{Syzygium}

\textit{Syzygium} is not mentioned in the traditional combinations
used to treat AD
from Japan, China and Korea.
However, it was found that \textit{Syzygium aromaticum} (clove)
extract was more potent than \textit{Acorus calamus}
in some assays.
\cite{ขจร2001fungistatic}




\subsection{Citrus}

Different species for Citrus are used in traditional medicines
which are employed against AD.
Typically, the mesocarpum and epicarpum
are studied for their neuroprotective, antiinflammatory and antioxidant
flavonoids.
\cite{?}

Nobiletin-rich Citrus reticulata peels,
a kampo medicine for Alzheimer's disease: A case series
\cite{seki2013nobiletin}





\subsection{Galantamine}












\subsection{Panax}

Ginseng was part of a mixture that
helped mice with macrophage abnormalities
survive \textit{Candida} infection.
\cite{akagawa1996protection}







\subsection{Polygala}

% background
The roots of \textit{Polygala tenuifolia (Pt)} are used in Asian medicine
for diverse purposes, including neurological conditions such as epilepsy.

% human
Polygala, as part of KKT, was effective in a small pilot trial.
\cite{arai2015effectiveness}



% animal
\textit{Pt} is cholinergic and prevents brain
damage caused by glutamate, AB, D-galactose, or scopolamine,
improving cognitive function and promoting neurogenesis
in animal models of AD.
\cite{park2002novel,
zhang2006effect,
sun2007effect,
zeng2009effect,
wen2010study,
shi2011effects}
A saponin from \textit{Pt}
increased BDNF and enhanced cognition
in a scopolamine model.
\cite{xue2009polygalasaponin}
Crude aqueous extract of \textit{Pt} markedly
enhanced memory in aged animals,
reducing MDA, increasing CAT activity,
increasing SOD, inhibiting AChE and MAO.
\cite{li2014memory}
In another model, in addition to the cholinergic effects,
neurotrophic effects were observed
after oral administration of \textit{Pt} extract.
\cite{yabe1997enhancements}
\textit{Pt} improved behavioral disorders in an animal model
of brain damage.
\cite{chen2004effect}

% in vitro
Saponins from \textit{Pt} induced NGF \textit{in vitro}.
\cite{yabe2003induction}
\textit{Pt} strongly inhibited AB-induced cell damage \textit{in vitro}.
%Ginseng, Astragalus and Polygala may have complementary actions.
\cite{naito2006characterization}
Tenuigenin, extracted from \textit{Pt},
decreased secretion of AB in cultured cells.
\cite{jia2004tenuigenin}
\textit{Pt} had antiinflammatory effects in microglia.
\cite{cheong2011anti}
A closely related species, \textit{Polygala tricornis},
also has constituents which reduce
neuroinflammation \textit{in vitro}.
\cite{li2012anti}
\textit{Pt} produces norepinephrine reuptake inhibitors
with antidepressant potential.
\cite{cheng2006antidepressant}
The methanolic extract of the dried leaves of \textit{Polygala japonicum}
contained a variety of molecules capable of
inhibiting lipopolysaccharide-induced nitric oxide production in BV2 microglia
at the concentrations ranging from 1.0 to 100.0 μM.
\cite{kim2009chemical}


% toxicity











\subsection{Fungi}

Many mushrooms and other fungi are used around the world for medical purposes.
There is a robust scientific rationale suggesting that some
fungal products may prove to be effective treatments for AD.

\subsubsection{Ganoderma}

Aqueous extract of \textit{Ganoderma lucidum}
was able to prevent the harmful effects of
amyloid-$\beta$ in an \textit{in vitro} study,
preserving the synaptic density protein, synaptophysin
in a dose-dependent manner.
\cite{lai2008antagonizing}

\textit{Ganoderma lucidum} was able to prevent neurotoxicity and
hippocampal degeneration while improving cognitive dysfunction in an animal model
of AD produced by toxic oxidative stress via streptozotocin (STZ).
\cite{zhou2012neuroprotective}

\textit{Ganoderma lucidum} contains small molecules which inhibit AChE.
\cite{lee2011selective}

An immune-modulating polysaccharide from \textit{Ganoderma atrum}
was able to attenuate neuronal apoptosis
by preventing oxidative damage by modifying the
redox system and helping to maintain calcium homeostasis
in an animal model.
\cite{li2011ganoderma}

Learning and memory were improved with
administration of \textit{Ganoderma lucidum}
in senescent mice,
improving antioxidant status.
\cite{wang2004effects}


\subsubsection{Hericium (He)}

He contains compounds with
antibiotic, anticarcinogenic, antidiabetic, antifatigue,
antihypertensive, antihyperlipidemic, antisenescence,
cardioprotective, hepatoprotective, nephroprotective,
and neuroprotective properties.
\cite{friedman2015chemistry}

Oral \textit{Hericium erinaceus} fruit body
was clearly better than placebo
improving cognitive impairment
measured by the Revised Hasegawa Dementia Scale
in a double-blind clinical trial on 30 patients
with mild cognitive impairment.
\cite{mori2009improving}

% what is amyloban?
Amyloban(tm), a standardized extract of He,
performed admirably in an open label case series
of 10 patients with refractory schizophrenia.
% get paper
\cite{inanaga2014improvement}
Similarly, Amyloban(tm) restored cognitive function
in three patients.
\cite{inanaga2015treatment}

% fruit body
Preparations of the fruiting body of He
attenuated amyloid-$\beta$ damage in PC-12 cells.
\cite{liu2015systemic}


He extracts inhibited AChE and displayed antioxidative activity.
\cite{jung2007ache}



Daily oral administration of
aqueous extract of He mushroom
promoted peripheral nerve regeneration
in an animal model.
\cite{wong2011peripheral,
wong2014hericium}


% mycelium
Over 20 years ago, the erinacines,
strong inducers of nerve growth factor (NGF) were discovered
in the mycelium of He.
\cite{kawagishi1994erinacines}
These reults were confirmed in astroglial cells, \textit{in vitro}.
\cite{
kawagishi1996erinacines,
kawagishi1996erinacine}
At least 9 erinacines have been found with potent NGF-inducing
activities. Erinacine H at 33.3 $\mu$g/ml provoked 5 times
the normal NGF production of normal astroglial cells in culture.
\cite{
lee2000two,
kawagishi2006erinacines}
A small molecule, non-erinacine, phenolic compound derived from
the mycelium of He was noted as an antimicrobial.
\cite{okamoto1993antimicrobial}
Ethanol extract of He mycelium was able to
prevent PC-12 apoptosis via the ROS-caspase dependent pathway.
\cite{chang2016improvement}

\textit{Hericium ramosum} mycelium had potent antioxidant activity
and oral administration was able to
induce NGF in the hippocampus \textit{in vivo}.
\cite{suruga2015effects}

% safety
Erinacine A was found to be non-genotoxic and
non-mutagenic in a standard battery of assays.
\cite{li2014genotoxicity}
Erinacine A enriched He mycelium caused no side effects
in an animal model with 28 days of 3g/kg dosing.
\cite{li2014evaluation}








\subsubsection{Poria cocos \textit{Pc}}

Poria is the most common ingredient in the
many Asian mixtures which are used for AD.

In animal models Poria has been shown to be essential
in the action of the Chinese mixture Kaixinsan.
\cite{gao2010comparision}
% AB
Water extract of \textit{Pc} attenuated
amyloid-$\beta$-induced oxidative stress and apoptosis
\textit{in vitro}
\cite{park2009poria}
Poria is used in a Korean medicine, Jangwonhwan,
which shows promising preclinical results.
\cite{seo2010modified,
seo2010oriental}
Poria enhanced learning and memory in an animal
model of scopolamine-induced dysfunction
\cite{zhang2012effects}
and may have AChE-inhibiting activity
as it did in an animal model when administered with
\textit{Polygala}.
\cite{li2011experimental}

Decoction of \textit{Pc} improved learning and memory
in mice subjected to scopolamine and alcohol.
\cite{zhang2012effects}

Poria mitigated chronic kidney disease in an animal model.
\cite{zhao2013urinary}







\begin{table*}[htp]
\centering

\begin{tabular}{||c c c c c c c c||}
 \hline
                & A-$\beta$ & AF & AO & AI & AChE & NGF & AMPK \\
 \hline\hline
 Acorus         &           & XX &  X &    &   X  &     &      \\
 Berberine      &           &    &    & X  &   X  &     &  X   \\
 Curcuma        &     X     & X  & XX & XX &      &     &      \\
 Hericium       &           &    &  X &  X &      &  XX &      \\
 Polygala       &     X     & X  &  X &  X &   X  &     &      \\
 Poria          &     X     &    &    &    &      &     &      \\
 \hline
\end{tabular}
\caption{Genus to Effects}
\label{table:effects}
\end{table*}



%
%\subsection{Natural Products in Use Without Sufficient Evidence}
%Some NPs are used to treat AD without sufficient evidence that they would be safe and effective.


%\subsubsection{Angelica (Dong quai)}
% but .. it's in combinations ..


