\documentclass[twocolumn]{article}

% to include images
% \usepackage[dvips]{graphicx}
% to use PostScript fonts
% \usepackage{pslatex}

%\usepackage{cite}
%\usepackage[sorting=none]{biblatex}
%\bibliography{mybib}

\usepackage[colorlinks=true, linkcolor=blue]{hyperref}
\usepackage[hypcap]{caption}


\setlength{\parindent}{1em}
\setlength{\parskip}{1em}
\setlength{\columnsep}{1.5em}


\begin{document}
\onecolumn
\title{A Natural Product Combination for Alzheimer's Disease}
\author{Skylar Saveland}

\maketitle

\begin{abstract}


%(Motivation)
Alzheimer's disease (AD) is a chronic neurodegenerative condition
associated with old age. It is the most common cause of dementia,
inflicting an increasing social and economic burden.
Medications which are currently approved in the United States
do not reverse or prevent disease progression.
Therefore, new therapies with more dramatic impact are sought.

%(Natural Products)
There has been abundant interest in natural products to treat Alzheimer's.
A large number of preclinical studies have demonstrated
that many crude natural products have therapeutic potential.
%But, in some cases, preclinical results have failed to
%translate into significant findings in randomized,
%placebo-controlled clinical trials (RCTs).
%(Traditional Medicine)
Clinical trials have been conducted on a few of these natural products.
A number of traditional medicines which are combinations of natural products
have been used to treat AD.
Dozens of natural products in various combinations are regularly employed
around the world, particularly in China, Japan and Korea.
Some clinical trials
have been conducted that support the usefulness of these combinations.
Ascertaining which constituents are responsible for the positive effects,
what interactions may be involved,
and what role synergism plays, are daunting combinatorial problems.

%(AD Causation)
Prevailing theories of AD causation have focused on the observed
biochemical abnormalities of the senile brain.
Observations of amyloid-$\beta$ plaques, acetylcholine (AChE) deficits,
tau protein abnormalities,
neuroinflammation and oxidative stress,
and aberrant signalling cascades leading to neuronal apoptosis
have partially elucidated the clinical picture
but have failed to produce satisfactory pharmaceuticals.
Natural products are rich in substances which target
relevant pathways in AD.

Recent research has produced strong evidence
that AD may be caused by fungal infection in the central nervous system (CNS).
Many natural products which have been investigated for their potential
to curb the biochemical abnormalities associated with AD
also show promise as antifungal agents.

%(Summary)
Here we conduct a review of the available literature in an attempt
to determine which natural products are the most promising
candidates for AD therapy.
Preferred candidates will have
success in clinical trial,
clean safety profiles
and robust biochemical rationale.
We propose a combination of the best candidates to be tested in a new RCT.
\end{abstract}
\tableofcontents
\twocolumn

\section{Background}

% Natural medicines are important and viable
Natural products have probably been used as medicine for over 60,000 years.
From then until modern times,
more than 200,000 natural products and compounds have been discovered.
\cite{ji2009natural}
In recent history, modern chemistry has allowed the chemical design of
individual molecules for pharmaceutical applications.
But, applying individual molecules to chronic neurodegenerative diseases, for instance,
has brought scant success.
Interest in natural medicine has sustained through the modern era.
A majority of drugs derive from natural products.
Of 175 small molecules used against cancer up to 2010,
74.8\% were not completely synthetic and
48.6\%, were directly derived from natural products.
\cite{newman2012natural}
Antibiotic and antifungal compounds have come from leads provided by nature.
Antibiotic discovery has been dependent on metabolites produced by soil bacteria.
Willow bark, the source of aspirin, was used from ancient times.
\cite{laursen2004phenazine}
Ethnobotany still provides a rich resevoir of CNS-active pharmacological leads.
\cite{mcclatchey2009ethnobotany, perry1999medicinal}

% Combinations and synergism
Traditional medicines from China, Japan and China invariably employ combinations
of different materials from plants and fungi.
A large number of active components have been identified.
\cite{gao2013research}
Synergistic relationships have been shown between substances
in traditional combinations of natural agents.
Multiple active components working together may be the key
to future AD treatments.
\cite{kong2009hope, liu2014history}

% Alzheimer's in particular
Psycotropic medicines derived from natural products are of particularly promising.
\cite{lake2000psychotropic}
Significant evidence exists that suggests that NPs
may be effective psychotherapeutics.
\cite{fugh1999dietary}
A wealth of NPs are candidates for the treatment of Alzheimer's in particular.
\cite{houghton2005natural}
In addition to biochemical target activities,
traditional medicine can provide improvements in
cognitive impairments, energy/fatigue, mood, and anxiety.
\cite{divino2011role}



% Specific molecules, targets
Many acetylcholinesterase(ACHE) inhibitors with potential clinical relevance
have been discovered from natural products.
\cite{barbosa2006natural}

NPs may play a role in inhibiting microglial neurotoxicity.
\cite{choi2011inhibitors}

In addition to curcumin, turmeric contains additional small molecules
that were able to protect cell cultures from AB.
\cite{park2002discovery}



\section{Clinical Evidence for Combinations of Natural Products}

Many different traditional mixtures exist which are used
treat AD. The mixtures have many overlapping components.

Table \ref{table:mixtures} shows the degree of overlap between
individual natural products and some traditional mixtures.



\subsection{Traditions}
% Kampo
\subsubsection{Kampo}

Kampo medicine, a traditional medicine of Japan,
is a complex system of individualized diagnosis and treatment.
An important component of Kampo is precise formulations of natural products
based on older Chinese recipes.
148 or more combinations are covered by the national health insurance system in Japan.
Of 135 published randomized controlled trials on Kampo combinations
published between 1986 and 2007,
37 were available in English.
As chronic degenerative diseases have become more prominent in an aging population,
interest in Kampo has increased as has Kampo's integration into modern medical practice.
A recent survey indicated that 70\% of Japanese physicians prescribe Kampo formulations.
Insurance-coverd formulations include standardized extracts and crude decoctions.
Applying Kampo formulations to conventional diagnoses is a challenge because
in traditional Kampo, the same conventional diagnosis could result in the prescription
of different formulations while different conventional diagnoses could result
in the prescription of the same formulation.
\cite{watanabe2011traditional}
The testing specfic individual formulations on a specific conventional diagnosis
is the most useful for applying Kampo formulations in clinical practice in
countries like the United States where expert Kampo practioners are very few.
Therefore, for the purposed of this review, the best evidence will be considered
to be randomized, placebo-controlled clinical trials on specific formulations
for the specific diagnosis of AD.
Kampo may evolve based on new scientific evidence.
\cite{terasawa2004evidence}

\subsubsection{TCM}
\subsubsection{Mixtures Used in Other Countries}



\begin{table*}[t]
\centering

\begin{tabular}{||c c c c c c c c c c c c||}
 \hline
 Genus          & YKS & DTD & FMJ & PN-1 & SZL & WD & BDW & CMT & HCKT & KRBT & CTS \\
 \hline\hline
 %              &     &     &     &      &     &    &     &     &      &      &     \\
 % Fungi
 Poria          &  X  &  X  &  X  &      &  X  & X  &  X  &  X  &      &      &  X  \\
 % Asteraceae
 Atractylodis   &  X  &     &     &      &     &    &     &     &  X   &      &     \\
 % Apiaceae
 Ligusticum     &  X  &     &     &      &     &    &     &     &      &      &     \\
 Angelica       &  X  &     &     &      &     &    &     &     &  X   &      &     \\
 Bupleurum      &  X  &     &     &      &     &    &     &     &  X   &      &     \\
 % Fabaceae
 Glycyrrhiza    &  X  &     &     &      &     &    &     &     &  X   &  X   &  X  \\
 Astragalus     &     &     &     &  X   &     &    &     &     &  X   &      &     \\
 % Rubiaceae
 Uncariae       &  X  &     &     &      &     &    &     &     &      &      &  X  \\
 % Araceae
 Arisaema       &     &  X  &     &      &     &    &     &     &      &      &     \\
 Pinellia       &     &  X  &     &      &     & X  &     &  X  &      &      &     \\
 % Rutaceae
 Citrus         &     &  X  &  X  &      &     & X  &     &     &  X   &      &  X  \\
 % Acoraceae (formerly Araceae)
 Acorus         &     &  X  &  X  &      &  X  & X  &     &  X  &      &      &     \\
 % Araliaceae
 Ginseng        &     &  X  &     &      &     &    &     &     &  X   &      &  X  \\
 % Poaceae
 Bambusa        &     &  X  &     &      &     & X  &     &     &      &      &     \\
 % Zingiberaceae
 Zingiberis     &     &  X  &     &      &     &    &     &     &  X   &  X   &  X  \\
 % Orobanchaceae
 Rehmannia      &     &     &  X  &      &     &    &  X  &     &      &      &     \\
 % Asparagaceae
 Ophiopogon     &     &     &  X  &      &     &    &     &     &      &      &  X  \\
 Anemarrhena    &     &     &  X  &      &     &    &     &     &      &      &     \\
 % Paeoniaceae
 Paeonia        &     &     &  X  &      &     &    &  X  &     &      &  X   &     \\
 % Orchidaceae
 Dendrobium     &     &     &  X  &      &     &    &     &     &      &      &     \\
 % Lardizabalaceaer
 Akebia         &     &     &  X  &      &     &    &     &     &      &      &     \\
 % Orobanchaceae
 Cistanche      &     &     &     &   X  &     &    &     &     &      &      &     \\
 % Campanulaceae
 Codonopsis     &     &     &     &      &  X  &    &     &     &      &      &     \\
 % Lauraceae
 Cinnamomum     &     &     &     &      &  X  &    &  X  &     &      &   X  &     \\
 % Polygalaceae
 Polygala       &     &     &     &      &  X  &    &     &  X  &      &      &     \\

 % note: SZL and KRBT contain fossil bones, Ossa Draconis


% &     &     &     &      &     &    &     &     &      &      &     \\
% &     &     &     &      &     &    &     &     &      &      &     \\
% &     &     &     &      &     &    &     &     &      &      &     \\
% &     &     &     &      &     &    &     &     &      &      &     \\

 % Ranunculaceae
 Aconitum       &     &     &     &      &     &    &  X  &     &      &      &     \\


 %Astragalus     & 6   &     &&&&&&&&&&& \\
 %5 & 88 & 788 & 6344 \\ [1ex]
 \hline
\end{tabular}
\caption{Mixture to Genus Matrix}
\label{table:mixtures}
\end{table*}


% Combinations

% Yokukansan
\subsection{Yokukansan (YKS)}
Yokukansan is a mixture of seven crude natural products,
Atractylodis lanceae rhizoma,
Poria cocos sclerotia,
Cnidii rhizoma,
Angelicae radix,
Bupleuri radix,
Glycyrrhizae radix,
and Uncariae uncis cum ramulus.
Yokukansan is called Yi-Gan San in TCM.
\cite{iwasaki2005randomized}
Clinical trials have demonstrated Yokukan-san’s efficacy
in treating patients with BPSD. Accordingly,
Yokukansan has been listed by The Japanese Society of Neurology
in the Japanese Guidelines for the Management of Dementia
since 2010.
A recent review found 13 clinical trials of varying quality
included a total of 466 patients that found YKS to be a safe
and effective way to raise Neuropsychiatric Inventory (NPI) scores.
The Mini-Mental State Examination (MMSE) score of cognitive impairment
and the Disability Assessment for Dementia (DAD) score of caregiver burden
were unimproved, however.
\cite{okamoto2014yokukan}
Another review of noted improvements in the activities of daily living (ADL) score.
\cite{matsuda2013yokukansan}
Repeated clinical trials have proven the efficacy of YKS in improving
NPI and ADL scores in patients with AD.
\cite{mizukami2014kampo}

In a cross-over study of 106 patients to investigate the use of Yokukansan
to treat the behavioural and psychological symptoms of dementia (BPSD),
a significant improvement in
Neuropsychiatric Inventory (NPI) was found.
Significant improvements were observed in delusions,
hallucinations, agitation/aggression, depression, anxiety, and irritability/lability.
Effects were sustained for one month after treatment was ceased.
Cognitive function was not significantly improved.
\cite{mizukami2009randomized}
In a 52 patient RCT of Yokukansan to treat dementia,
BPSD and ADL scores were improved.
\cite{iwasaki2005randomized}
In an open-label study of 26 patients who received
7.5 grams/day of Yokukansan for 4 weeks,
success was seen in reducing
hallucinations, agitation, anxiety, irritability
and abnormal behavior.
But, overall disability and congnitive function were not improved.
The mixture was well tolerated.
\cite{hayashi2010treatment}


Yokukansan is considered very safe.
In a case series of 3 patients between 10 and 13 years old,
YKS considered effective in treating pediatric emtional and behavioral problems
within 14-21 days.
\cite{tanaka2013potential}



\subsection{Chotosan}
Chotosan is a mixture of 11 natural products,
Uncariae Uncis cum Ramulus,
Aurantii Nobilis pericarpium,
Pinelliae tuber,
Ophiopogonis tuber,
Poria cocos,
Ginseng radix,
Saposhnikoviae radix,
Chrysanthemi flos,
Glycyrrhizae radix,
Zingiberis rhizome,
and Gypsum fibrosum.
Along with YKS, Chotosan has shown promise in clinical and
preclinical studies as a treatment for AD.
Both mixtures contain Uncariae Uncis.
\cite{matsumoto2013kampo}


\subsection{Keishi-ka-ryukotsu-borei-to (KRBT)}

KRBT is a mixture of 7 natural products:
cinnamon bark,
peony root,
jujube fruit,
oyster shell,
fossilized bone,
glycyrrhiza,
and ginger rhizome.
That was found to effectively BPSD in a case report.
Gonadotrophin profiles were positively altered.
\cite{niitsu2013behavioural}

\subsection{Kamikohito (KKT)}

An animal model revealed that KKT
improved amyloid-$\beta$-induced tau phosphorylation and axonal atrophy
even after axonal degeneration had progressed.
\cite{watari2014new, watari2015comparing}




\subsection{Ninjin'yoeito (NYT)}


23 patients who had a insufficient response to donepezil
received donepezil alone or donepezil and NYT.
A 2-year follow-up showed that patients receiving NYT
had an improved cognitive outcome and alleviation of AD-related depression.
\cite{kudoh2015effect}





\subsection{Hochuekkito}
Hochuekkito is a mix of 10 natural products,
Astragali Radix,
Atractylodis lanceae Rhizoma,
Ginseng Radix,
Angelicae Radix,
Bupleuri Radix,
Zizyphi Fructus,
Aurantii Bobilis Pericarpium,
Glycyrrhizae Radix,
Cimicifugae Rhizoma,
and Zingiberis Rhizoma.
\cite{kiyohara2011polysaccharide}

In a placebo-controlled clinical trial of Hochuekkito
showed that the formulation improved the QOL and immunological status
of elderly patients with weakness.
\cite{satoh2005randomized}

\subsection{Zokumei-to (ZMT)}

In an animal model of AD using Amyloid-$\beta$,
ZMT treatment significantly increased the level of expression of
synaptophysin up to the control level.
Memory impairment and synaptic loss was ameliorated in the mice
even after impairment had progressed.
\cite{tohda2003repair}



\subsection{Ba Wei Di Huang Wan (BWD)}
BWD is a traditional Chinese formulation of 8 natural products,
Rehmannia glutinosa,
Cornus officinalis,
Dioscoreabatatas root,
Alisma orientale rhizome,
Poria cocos,
Paeonia suffruti-cosa,
Cinnamomum cassia,
and Aconitum carmichaeli.

In a placebo-controlled RCT of 33 patients with AD,
cognitive function was significantly improved by BWD compared to placebo
based on the Mini-Mental State Examination (MMSE).
The activities of daily living (ADLs) score was also improved
versus placebo.
Scores returned to baseline after eight weeks.
\cite{iwasaki2004randomized}


%
% Studies on individual constituents
%

% Individual Natural Products
\section{Clinical Evidence for Individual Natural Products}


In addition to the traditional formulations,
individual constituents are studied for their potential
to treat AD including
Huperzine,
Gingko,
Curcuma,
Salidroside,
Periwinkle-vinca,
Centella,
Melissa,
Polygala,
Salvia,
and Withania.
\cite{sun2013traditional}
Investigation in the West has been limited to a few herbs,
notably Gingko. Some herbs used in traditional European medicine
such as \textit{Salvia officinalis} and \textit{Melissa officinalis}
are under investigation for their potential as AD treatments.
\cite{perry1999medicinal}


\subsection{Salvia}

Salvia lavandulaefolia (Spanish sage)
had strong in vitro antiinflammatory activity relevant to AD.
\cite{perry2001vitro}
Salvia is a large genus with antifungal properties.
\cite{yuce2014essential, tabanca2006chemical}





Nobiletin-rich Citrus reticulata peels,
a kampo medicine for Alzheimer's disease: A case series
\cite{seki2013nobiletin}





% Biochemical Targets
\section{Biochemical Targets}


% fungal infection
\subsection{Fungal Infection}

Evidence has been put forth that fungal infection is detectable
in brain samples from Alzheimer's disease patients.
\cite{alonso2013fungal}
Misfolded amyloid proteins may originate from a transmissible infectious process
(like prion disease).
Some proteins behave as prions in yeast and other fungi.
\cite{soto2006amyloids}
Neuroinflammation is associated with innate immunity.
\cite{hauwel2005innate}
Fungal infection could explain the connection between
neuroinflammation, innate immunity and the toxic cell
debris associated with AD.
Amyloid-$\beta$ is toxic to yeast cells
as well as nerve cells.
A yeast model of amyloid toxicity has been proposed
because of yeast's susceptibility to the protein.
\cite{treusch2011functional}





Resveratrol is a promising neuroprotective substance.
\cite{li2012resveratrol}


\subsection{Amyloid-$\beta$}


An animal model
\cite{tohda2003repair}


\subsection{Inflammation}


Long-term use of Non-Steroidal Antiinflammatory Drugs (NSAIDs)
have been associated with a dramatic decrease in the risk of AD
in a large cohort.
The risk ratio was 0.20 (95 percent confidence interval, 0.05 to 0.83).
\cite{in2001nonsteroidal}

Other epidemiological evidence suggests that individuals
with diets right in antiinflammatory Curmcumin
are at lower risk of developing AD.
\cite{?}


Inflammation may be the key to initiating the toxic cascade that
characterized Alzheimer's rather than the result of
midfolded proteins such as amyloid and tau.
\cite{heneka2007inflammatory}







\subsection{Cholinergic Drugs}

Cholinergic drugs were able to slow the progression of AD
after 2 years of follow-up.
Although, cognitive deterioration was not stopped.
\cite{requena2006effects}





% Challenges
\section{Challenges}

A major concern raised about crude natural products is
the challenge of standardization.


A formulation called kamikihito (KKT) in Japan
and kami-guibi-tang (KGT) in Korea
were comparable in enhancing memory in an animal model,
although chromatography revealed differences.
\cite{watari2015comparing}



% Proposed Formulation


% Conclusion
\section{Conclusion}
Write your conclusion here.










\bibliographystyle{unsrt}
\bibliography{mybib}
%\printbibliography

\end{document}



