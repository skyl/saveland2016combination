\documentclass[twocolumn]{article}

% to include images
% \usepackage[dvips]{graphicx}
% to use PostScript fonts
% \usepackage{pslatex}

%\usepackage{cite}
%\usepackage[sorting=none]{biblatex}
%\bibliography{mybib}

\usepackage[colorlinks=true, linkcolor=blue]{hyperref}
\usepackage[hypcap]{caption}


\setlength{\parindent}{1em}
\setlength{\parskip}{1em}
\setlength{\columnsep}{1.5em}


\begin{document}
\onecolumn
\title{A Natural Product Combination for Alzheimer's Disease}
\author{Skylar Saveland}

\maketitle

\begin{abstract}


%(Motivation)
Alzheimer's disease (AD) is a chronic neurodegenerative condition
associated with old age. It is the most common cause of dementia,
inflicting an increasing social and economic burden.
Medications which are currently approved in the United States
do not reverse or prevent disease progression.
Therefore, new therapies with more dramatic impact are sought.

%(Natural Products)
There has been abundant interest in natural products to treat Alzheimer's.
A large number of preclinical studies have demonstrated
that many crude natural products have therapeutic potential.
%But, in some cases, preclinical results have failed to
%translate into significant findings in randomized,
%placebo-controlled clinical trials (RCTs).
%(Traditional Medicine)
Clinical trials have been conducted on a few of these natural products.
A number of traditional medicines which are combinations of natural products
have been used to treat AD.
Dozens of natural products in various combinations are regularly employed
around the world, particularly in China, Japan and Korea.
Some clinical trials
have been conducted that support the usefulness of these combinations.
Ascertaining which constituents are responsible for the positive effects,
what interactions may be involved,
and what role synergism plays, are daunting combinatorial problems.

%(AD Causation)
Prevailing theories of AD causation have focused on the observed
biochemical abnormalities of the senile brain.
Observations of amyloid-$\beta$ plaques, acetylcholine (AChE) deficits,
tau protein abnormalities,
neuroinflammation and oxidative stress,
and aberrant signalling cascades leading to neuronal apoptosis
have partially elucidated the clinical picture
but have failed to produce satisfactory pharmaceuticals.
Natural products are rich in substances which target
relevant pathways in AD.

Recent research has produced strong evidence
that AD may be caused by fungal infection in the central nervous system (CNS).
Many natural products which have been investigated for their potential
to curb the biochemical abnormalities associated with AD
also show promise as antifungal agents.

%(Summary)
Here we conduct a review of the available literature in an attempt
to determine which natural products are the most promising
candidates for AD therapy.
Preferred candidates will have
success in clinical trial,
clean safety profiles
and robust biochemical rationale.
We propose a combination of the best candidates to be tested in a new RCT.
\end{abstract}
\tableofcontents
\twocolumn

\section{Background}

% Natural medicines are important and viable
Natural products have probably been used as medicine for over 60,000 years.
From then until modern times,
more than 200,000 natural products and compounds have been discovered.
\cite{ji2009natural}
In recent history, modern chemistry has allowed the chemical design of
individual molecules for pharmaceutical applications.
But, applying individual molecules to chronic neurodegenerative diseases, for instance,
has brought scant success.
Interest in natural medicine has sustained through the modern era.
A majority of drugs derive from natural products.
Of 175 small molecules used against cancer up to 2010,
74.8\% were not completely synthetic and
48.6\%, were directly derived from natural products.
\cite{newman2012natural}
Antibiotic and antifungal compounds have come from leads provided by nature.
Antibiotic discovery has been dependent on metabolites produced by soil bacteria.
Willow bark, the source of aspirin, was used from ancient times.
\cite{laursen2004phenazine}
Ethnobotany still provides a rich resevoir of CNS-active pharmacological leads.
\cite{mcclatchey2009ethnobotany, perry1999medicinal}

% Combinations and synergism
Traditional medicines from China, Japan and Korea invariably employ combinations
of different materials from plants and fungi.
A large number of active components have been identified.
\cite{gao2013research}
Synergistic relationships have been shown between substances
in traditional combinations of natural agents.
Multiple active components working together may be the key
to future AD treatments.
\cite{kong2009hope, liu2014history}

% Alzheimer's in particular
Psycotropic medicines derived from natural products are particularly promising.
\cite{lake2000psychotropic}
Significant evidence exists that suggests that NPs
may be effective psychotherapeutics.
\cite{fugh1999dietary}
A wealth of NPs are candidates for the treatment of Alzheimer's in particular.
\cite{houghton2005natural}
In addition to biochemical target activities,
traditional medicine can provide improvements in
cognitive impairments, energy/fatigue, mood, and anxiety.
\cite{divino2011role}
% Specific molecules, targets
Many acetylcholinesterase(ACHE) inhibitors with potential clinical relevance
have been discovered from natural products.
\cite{barbosa2006natural}



\subsection{Conventional Understanding of Alzheimer's Dementia}

AD is a disease of old age
characterized by slow onset with a
progressive loss of cognitive function and gradual decline into dementia.
AD is the most common cause of dementia.
The time from diagnosis until death is usually 8-10 years.
The most notable pathological changes are plaques of misfolded proteins,
neurofibrillary tangles, alterations in the microvasculature of the brain,
oxidative stress.

\subsubsection{Amyloid-$\beta$ (AB)}

The cerebral deposition of amyloid-$\beta$ protein in AD patients
has been recognized for decades.
These proteins, which begin to accumulate
for years or even decades before the onset of dementia,
 have long been considered central to the pathology of AD.
\cite{citron1992mutation}

A popular hypothesis states that AB is the causative agent in AD,
causing the cascade of abnormality observed. Genetic mutations
which cause abnormal amyloid precursor protein (APP) are implicated.
AB is neurotoxic and could lead to the neurofibrillary tangles
and ultimately the death of neurons.
\cite{hardy1992alzheimer}


Soluble AB was strongly correlated with AD severity.
Measures of insoluble AB could distinguish AD patients from
controls; but, could not predict AD severity.
AB is usually thought of as extracellular.
However, soluble AB can be extracellular or intracellular.
\cite{mclean1999soluble}


More recently, synaptotoxic AB oligomers have been implicated
in AD biopathology,
triggering the accumulation of reactive oxygen species (ROSs).
This provides an explanation for the therapeutic action
of memantine.
\cite{klein2013synaptotoxic}

% ApoE
% Tau
% genetics


\subsubsection{Microcirculation}

A idea related to AB toxicity is that disturbed microcirculation in the brain
causes AD.
\cite{de1993can}
One recent review stated that the AB hypothesis of AD causation has
proven inadequate and that therapeutic strategies should
focus on the microcirculation hypothesis and supporting normal angiogenesis.
\cite{drachman2014amyloid}

The microcirculation hypthesis states that the primary cause of
AD is related to deficiencies in the microvasculature of the brain.
These deficits may allow the inflow of neurotoxins into the brain
and, perhaps more importantly, prevent the adequate outflows of
neurotoxins such as AB.

BBB dysfunction has been noted in AD patients, including leakage which may allow
neurotoxins and infectious agents to enter the brain.
However, this finding is controversial.
Some studies failed to find additional leakage in AD brains
when compared to age-matched controls.
Other BBB dysfunctions observed in AD involve inadequate nutrient supply and
inadequate clearance of toxic substances.
In addition, altered proteins at the neuro-vascular unit may promote
inflammation, oxidative stress and neuronal damage.
\cite{erickson2013blood}


\subsubsection{Inflammation}

Certainly AD pathology involves extracellular plaques involving
AB, neurofibrillary tangles with abnormal tau protiens,
vascular malfunction
and cell death caused by ROS and inflammation.
Some propose that inflammation and oxidative stress precede
the development of the AB plaques, initiating the
pathological cascade observed in AD.
\cite{luque2014oxidative}




\subsubsection{Infection}

The idea that infections cause AD has been widely studied and
has not been ruled out. In fact,
a substantial amount of evidence suggests that
AD may be caused by infection.
Histopathological hallmarks of AD are known to occur in
chronic infections, including those of the CNS.
\cite{
iked1995numerous,
mandybur1990distribution,
kueh1984amyloid,
liberski1994transmissible,
kobayashi2008plaque,
sikorska2009ultrastructural,
de1984serum,
looi1988immunohistochemical,
rocken1999generalized,
wangel1982family,
tank2000renal,
urban1993ct}
In addition, evidence has emerged that AB is antimicrobial
and may be produced as an innate immune response to infection.
% sweet paper.
\cite{soscia2010alzheimer}


Inflammatory cytokines typical of immune response to infection
are associated with the inflammation observed in AD.
Elevated levels of interleukin-1, interleukin-6 and
tumor necrosis factor-$\alpha$ have been observed.
\cite{
sastre2006contribution,
holmes2011systemic,
akiyama2000inflammation,
cojocaru2011study}

Pathogens most commonly studied in AD etiology have been
viral, bacterial and prional. These include
HSV-1,
\textit{C. pneumoniae},
\textit{B. burgdorferi},
\textit{H. pylori},
and prions.
One group has argues that HSV-1 initiates the pathological
cascade.
\cite{ball2013intracerebral}
Although, if HSV-1 is the primary cause of AD,
how do some people without HSV-1 infection develop AD?
Another group found that anti-HSV IgM was associated with
double the risk of developing AD, indicating that
reactivated HSV-1 infection increases the risk of developing AD.
\cite{lovheim2014reactivated}

The possibility of fungal pathogenisis
has been almost completely overlooked.
\cite{mawanda2013can}

Recently, a series of studies from the same research
group have provided evidence that AD may be caused
by fungal infection in the CNS
\cite{
pisa2015different,
alonso2013fungal,
pisa2015direct
}




\subsubsection{Other theories of causation}

Deficits in glucose metabolism including
insulin resistance have been implicated as
risk factors for AD.
\cite{erickson2013blood}
Some have even proposed that
these irregularities could play a role in
the etiology of the disease.
\cite{luque2014oxidative}
Research has indicated that those who develop
type 2 diabetes and those who develop AD
share some genetic risk factors.
\cite{hao2015shared}


One review found that AD may be related to sleep apnea.
\cite{pan2014can}


% full subsection for genetics?
In addition to a number of possible environmental hazards,
an epidemiological review found a number of genes which
increased the risk of early-onset and late-onset AD.
\cite{jiang2013epidemiology}




\subsection{Traditions}
% Kampo
\subsubsection{Kampo}

Kampo medicine, a traditional medicine of Japan,
is a complex system of individualized diagnosis and treatment.
An important component of Kampo is precise formulations of natural products
based on older Chinese recipes.
148 or more combinations are covered by the national health insurance system in Japan.
Of 135 published randomized controlled trials on Kampo combinations
published between 1986 and 2007,
37 were available in English.
As chronic degenerative diseases have become more prominent in an aging population,
interest in Kampo has increased as has Kampo's integration into modern medical practice.
A recent survey indicated that 70\% of Japanese physicians prescribe Kampo formulations.
% About 2% of prescriptions are Kampo /citation? ~ 200,000 a year
Insurance-covered formulations include standardized extracts and crude decoctions.
Applying Kampo formulations to conventional diagnoses is a challenge because
in traditional Kampo, the same conventional diagnosis could result in the prescription
of different formulations while different conventional diagnoses could result
in the prescription of the same formulation.
\cite{watanabe2011traditional}
However, Kampo may evolve based on new scientific evidence.
\cite{terasawa2004evidence}

The testing specfic individual formulations on a specific conventional diagnosis
is useful for applying Kampo formulations in clinical practice in
countries like the United States where expert Kampo practioners are few.
Therefore, for the purposed of this review, the best evidence will be considered
to be randomized, placebo-controlled clinical trials on specific formulations
for the specific diagnosis of AD.

\subsubsection{TCM}

A review of Chinese Herbal Medicine for the management of vascular dementia (VD),
a disease similar to AD,
was conducted, highlighting the urgent need for more clinical trials of high quality.
% long, detailed PhD thesis from Australia
\cite{liu2008development}

\subsubsection{Mixtures Used in Other Countries}










% TODO: move
%
%NPs may play a role in inhibiting microglial neurotoxicity.
%\cite{choi2011inhibitors}

%In addition to curcumin, turmeric contains additional small molecules
%that were able to protect cell cultures from AB.
%\cite{park2002discovery}



\section{Evidence for Combinations of Natural Products}

Many different traditional mixtures exist which are used
treat AD. The mixtures have many overlapping components.

Table \ref{table:mixtures} shows the degree of overlap between
individual natural products and some traditional mixtures.






\begin{table*}[htp]
\centering

\begin{tabular}{||c c c c c c c c c c c c||}
 \hline
 Genus          & YKS & DTD & FMJ & PN-1 & SZL & WD & BDW & CMT & HCKT & KRBT & CTS \\
 \hline\hline
 %              &     &     &     &      &     &    &     &     &      &      &     \\
 % Fungi
 Poria          &  X  &  X  &  X  &      &  X  & X  &  X  &  X  &      &      &  X  \\
 % Asteraceae
 Atractylodis   &  X  &     &     &      &     &    &     &     &  X   &      &     \\
 % Apiaceae
 Ligusticum     &  X  &     &     &      &     &    &     &     &      &      &     \\
 Angelica       &  X  &     &     &      &     &    &     &     &  X   &      &     \\
 Bupleurum      &  X  &     &     &      &     &    &     &     &  X   &      &     \\
 % Fabaceae
 Glycyrrhiza    &  X  &     &     &      &     &    &     &     &  X   &  X   &  X  \\
 Astragalus     &     &     &     &  X   &     &    &     &     &  X   &      &     \\
 % Rubiaceae
 Uncariae       &  X  &     &     &      &     &    &     &     &      &      &  X  \\
 % Araceae
 Arisaema       &     &  X  &     &      &     &    &     &     &      &      &     \\
 Pinellia       &     &  X  &     &      &     & X  &     &  X  &      &      &     \\
 % Rutaceae
 Citrus         &     &  X  &  X  &      &     & X  &     &     &  X   &      &  X  \\
 % Acoraceae (formerly Araceae)
 Acorus         &     &  X  &  X  &      &  X  & X  &     &  X  &      &      &     \\
 % Araliaceae
 Ginseng        &     &  X  &     &      &     &    &     &     &  X   &      &  X  \\
 % Poaceae
 Bambusa        &     &  X  &     &      &     & X  &     &     &      &      &     \\
 % Zingiberaceae
 Zingiberis     &     &  X  &     &      &     &    &     &     &  X   &  X   &  X  \\
 % Orobanchaceae
 Rehmannia      &     &     &  X  &      &     &    &  X  &     &      &      &     \\
 Gastrodia      &     &     &     &      &     & X  &     &     &      &      &     \\
 % Asparagaceae
 Ophiopogon     &     &     &  X  &      &     &    &     &     &      &      &  X  \\
 Anemarrhena    &     &     &  X  &      &     &    &     &     &      &      &     \\
 % Paeoniaceae
 Paeonia        &     &     &  X  &      &     &    &  X  &     &      &  X   &     \\
 % Orchidaceae
 Dendrobium     &     &     &  X  &      &     &    &     &     &      &      &     \\
 % Lardizabalaceaer
 Akebia         &     &     &  X  &      &     &    &     &     &      &      &     \\
 % Orobanchaceae
 Cistanche      &     &     &     &   X  &     &    &     &     &      &      &     \\
 % Campanulaceae
 Codonopsis     &     &     &     &      &  X  &    &     &     &      &      &     \\
 % Lauraceae
 Cinnamomum     &     &     &     &      &  X  &    &  X  &     &      &   X  &     \\
 % Polygalaceae
 Polygala       &     &     &     &      &  X  &    &     &  X  &      &      &     \\

 % note: SZL and KRBT contain fossil bones, Ossa Draconis
 % Oyster shell ...
 Ostrea         &     &     &     &      &     &    &     &     &      &   X  &     \\
 % Polygonaceae
 Polygonum      &     &     &     &      &     & X  &     &     &      &      &     \\
 % Rhamnaceae - jujube
 Ziziphus       &     &     &     &      &     & X  &     &     &  X   &   X  &     \\





% &     &     &     &      &     &    &     &     &      &      &     \\
% &     &     &     &      &     &    &     &     &      &      &     \\
% &     &     &     &      &     &    &     &     &      &      &     \\
% &     &     &     &      &     &    &     &     &      &      &     \\

 % Ranunculaceae
 Aconitum       &     &     &     &      &     &    &  X  &     &      &      &     \\


 %Astragalus     & 6   &     &&&&&&&&&&& \\
 %5 & 88 & 788 & 6344 \\ [1ex]
 \hline
\end{tabular}
\caption{Mixture to Genus Matrix}
\label{table:mixtures}
\end{table*}


% Combinations

% Yokukansan
\subsection{Yokukansan (YKS)}
Yokukansan is a mixture of seven crude natural products,
Atractylodis lanceae rhizoma,
Poria cocos sclerotia,
Cnidii rhizoma,
Angelicae radix,
Bupleuri radix,
Glycyrrhizae radix,
and Uncariae uncis cum ramulus.
Yokukansan is called Yi-Gan San in TCM.
\cite{iwasaki2005randomized}
Clinical trials have demonstrated Yokukan-san’s efficacy
in treating patients with BPSD. Accordingly,
Yokukansan has been listed by The Japanese Society of Neurology
in the Japanese Guidelines for the Management of Dementia
since 2010.
A recent review found 13 clinical trials of varying quality
included a total of 466 patients that found YKS to be a safe
and effective way to raise Neuropsychiatric Inventory (NPI) scores.
The Mini-Mental State Examination (MMSE) score of cognitive impairment
and the Disability Assessment for Dementia (DAD) score of caregiver burden
were unimproved, however.
\cite{okamoto2014yokukan}
Another review of noted improvements in the activities of daily living (ADL) score.
\cite{matsuda2013yokukansan}
Repeated clinical trials have proven the efficacy of YKS in improving
NPI and ADL scores in patients with AD.
\cite{mizukami2014kampo}

In a cross-over study of 106 patients to investigate the use of Yokukansan
to treat the behavioural and psychological symptoms of dementia (BPSD),
a significant improvement in
Neuropsychiatric Inventory (NPI) was found.
Significant improvements were observed in delusions,
hallucinations, agitation/aggression, depression, anxiety, and irritability/lability.
Effects were sustained for one month after treatment was ceased.
Cognitive function was not significantly improved.
\cite{mizukami2009randomized}
In a 52 patient RCT of Yokukansan to treat dementia,
BPSD and ADL scores were improved.
\cite{iwasaki2005randomized}
In an open-label study of 26 patients who received
7.5 grams/day of Yokukansan for 4 weeks,
success was seen in reducing
hallucinations, agitation, anxiety, irritability
and abnormal behavior.
But, overall disability and congnitive function were not improved.
The mixture was well tolerated.
\cite{hayashi2010treatment}


Yokukansan is considered very safe.
In a case series of 3 patients between 10 and 13 years old,
YKS considered effective in treating pediatric emotional and behavioral problems
within 14-21 days.
\cite{tanaka2013potential}


\subsection{Di-Tan Decoction (DTD)}
DTD is a combination of ...

A double-blind, randomized, placebo-controlled, add-on trial
testing the efficacy of DTD to treat cognitive impairment
in AD patients.
\cite{chua2015efficacy}



\subsection{Keishi-ka-ryukotsu-borei-to (KRBT)}

KRBT is a mixture of 7 natural products:
cinnamon bark,
peony root,
jujube fruit,
oyster shell,
fossilized bone,
glycyrrhiza,
and ginger rhizome
that was found to effectively BPSD in a case report.
Gonadotrophin profiles were positively altered.
\cite{niitsu2013behavioural}




\subsection{Chotosan}
Chotosan is a mixture of 11 natural products,
Uncariae Uncis cum Ramulus,
Aurantii Nobilis pericarpium,
Pinelliae tuber,
Ophiopogonis tuber,
Poria cocos,
Ginseng radix,
Saposhnikoviae radix,
Chrysanthemi flos,
Glycyrrhizae radix,
Zingiberis rhizome,
and Gypsum fibrosum.
Along with YKS, Chotosan has shown promise in clinical and
preclinical studies as a treatment for AD.
Both mixtures contain Uncariae Uncis.
\cite{matsumoto2013kampo}



\subsection{Kamikohito (KKT)}
KKT is composed of 14 crude drugs:

Ginseng Radix (P. ginseng C.A. Meyer),
Polygalae Radix (P. tenuifolia Willd.),
Astragali Radix (A. membranaceus Bunge),
Zizyphi Fructus (Zizyphus jujube Mill. var. inermis Rehd.),
Zizyphi Spinosi Semen (Z. jujube Mill. var. spinosa),
Angelicae Radix (Angelica acutiloba Kitagawa),
Glycyrrhizae Radix (Glycyrrhiza uralensis Fisch),
Atractylodis Rhizoma (Atractylodes japonica Koidzumi ex Kitamura),
Zingiberis Rhizoma (Zingiber officinale Roscoe),
Poria (Poria cocos Wolf),
Saussureae Radix (Saussurea lappa Clarke),
Longanae Arillus (Dimocarpus longana),
Bupleuri Radix (Bupleurum falcatum Linne), and
Gardeniae Fructus (Gardenia jasminoides Ellis).
Eleven of the fourteen NPs are listed in \ref{table:mixtures}.
Saussurea, Dimocarpus, and Gardenia are unique to KKT.

An animal model revealed that KKT
improved amyloid-$\beta$-induced tau phosphorylation and axonal atrophy
even after axonal degeneration had progressed.
\cite{watari2014new, watari2015comparing}




\subsection{Ninjin'yoeito (NYT)}


23 patients who had a insufficient response to donepezil
received donepezil alone or donepezil and NYT.
A 2-year follow-up showed that patients receiving NYT
had an improved cognitive outcome and alleviation of AD-related depression.
\cite{kudoh2015effect}





\subsection{Hochuekkito}
Hochuekkito is a mix of 10 natural products,
Astragali Radix,
Atractylodis lanceae Rhizoma,
Ginseng Radix,
Angelicae Radix,
Bupleuri Radix,
Zizyphi Fructus,
Aurantii Bobilis Pericarpium,
Glycyrrhizae Radix,
Cimicifugae Rhizoma,
and Zingiberis Rhizoma.
\cite{kiyohara2011polysaccharide}

In a placebo-controlled clinical trial of Hochuekkito
showed that the formulation improved the QOL and immunological status
of elderly patients with weakness.
\cite{satoh2005randomized}

\subsection{Zokumei-to (ZMT)}

In an animal model of AD using Amyloid-$\beta$,
ZMT treatment significantly increased the level of expression of
synaptophysin up to the control level.
Memory impairment and synaptic loss was ameliorated in the mice
even after impairment had progressed.
\cite{tohda2003repair}



\subsection{Ba Wei Di Huang Wan (BWD)}
BWD is a traditional Chinese formulation of 8 natural products,
Rehmannia glutinosa,
Cornus officinalis,
Dioscoreabatatas root,
Alisma orientale rhizome,
Poria cocos,
Paeonia suffruti-cosa,
Cinnamomum cassia,
and Aconitum carmichaeli.

In a placebo-controlled RCT of 33 patients with AD,
cognitive function was significantly improved by BWD compared to placebo
based on the Mini-Mental State Examination (MMSE).
The activities of daily living (ADLs) score was also improved
versus placebo.
Scores returned to baseline after eight weeks.
\cite{iwasaki2004randomized}




\subsection{Yishen Huazhuo decoction (YHD)}
YHD was found to be comparable or better than
the conventional AChE drug donepezil

\cite{zhang2015cognitive}


%
% Studies on individual constituents
%

% Individual Natural Products
\section{Evidence for Individual Natural Products}


In addition to the traditional formulations,
individual constituents are studied for their potential
to treat AD including
Huperzine,
Gingko,
Curcuma,
Salidroside,
Periwinkle-vinca,
Centella,
Melissa,
Polygala,
Salvia,
and Withania.
\cite{sun2013traditional}
Investigation in the West has been limited to a few herbs,
notably Gingko. Some herbs used in traditional European medicine
such as \textit{Salvia officinalis} and \textit{Melissa officinalis}
are under investigation for their potential as AD treatments.
\cite{perry1999medicinal}


\subsection{Curcuma}

Turmeric, \textit{Curcuma longa} root, has a long history of use as medicine.
\cite{?}
Curry consumption in old age may benefit cognitive function
according to epidemiological studies.
\cite{?}

Turmeric extract was effective in reducing the burden
of AB and tau in a transgenic mouse model of AD.
\cite{shytle2012optimized}

A case series reported that turmeric was remarkably effective
in treating 3 patients with AD.
\cite{hishikawa2012effects}



Curcumin, a phenolic compound derived from turmeric,
has been effective in animal models of AD.
\cite{?}
The effects are hypothesized to be antioxidant, antiinflammatory,
acetylcholinesterase-inhibiting,
and AB inflammation-inhibiting.
\cite{?}
\textit{In vivo} studies have show that
the bioavailability of curcumin alone is poor.
\cite{?}
However, it has been show that curcumin can pass the BBB.
Also, some compounds may be able to increase the bioavailability
of curcumin. In one study, piperine, a phenol(?) from black pepper
was able to increase curcumin concentrations in serum(?) by 20 times.
\cite{?}
Curcumin's success in animal models of AD is impressive.
\cite{?}


At least 4 clinical trials have been initiated
to test the efficacy of curcumin in treating AD.
No trial has been done on a crude extract of turmeric.(?)
Human safety trials have shown that curcumin is relatively safe
up to 12 grams a day.
However, a review of clinical trials evaluating curcumin
in AD have failed to establish curcumin as an effective drug.
\cite{hamaguchi2010review}

Compounds other than curcumin may be important to
prevent cognitive decline in the elderly
and even to help cure AD.















\subsection{Salvia}

Salvia lavandulaefolia (Spanish sage)
had strong in vitro antiinflammatory activity relevant to AD.
\cite{perry2001vitro}
Salvia is a large genus with antifungal properties.
\cite{yuce2014essential, tabanca2006chemical}



\subsection{Citrus}

Different species for Citrus are used in traditional medicines
which are employed against AD.
Typically, the mesocarpum and epicarpum
are studied for their neuroprotective, antiinflammatory and antioxidant
flavonoids.
\cite{?}

Nobiletin-rich Citrus reticulata peels,
a kampo medicine for Alzheimer's disease: A case series
\cite{seki2013nobiletin}


\subsection{Fungi}

Many mushrooms and other fungi are used around the world for medical purposes.
There is a robust scientific rationale suggesting that some
fungal products may prove to be effective treatments for AD.

\subsubsection{Ganoderma}

Aqueous extract of \textit{Ganoderma lucidum}
was able to prevent the harmful effects of
amyloid-$\beta$ in an \textit{in vitro} study,
preserving the synaptic density protein, synaptophysin
in a dose-dependent manner.
\cite{lai2008antagonizing}

\textit{Ganoderma lucidum} was able to prevent neurotoxicity and
hippocampal degeneration while improving cognitive dysfunction in an animal model
of AD produced by toxic oxidative stress via streptozotocin (STZ).
\cite{zhou2012neuroprotective}

\textit{Ganoderma lucidum} contains small molecules which inhibit AChE.
\cite{lee2011selective}

An immune-modulating polysaccharide from \textit{Ganoderma atrum}
was able to attenuate neuronal apoptosis
by preventing oxidative damage by modifying the
redox system and helping to maintain calcium homeostasis
in an animal model.
\cite{li2011ganoderma}

Learning and memory were improved with
administration of \textit{Ganoderma lucidum}
in senescent mice,
improving antioxidant status.
\cite{wang2004effects}


\subsubsection{Hericium}






% Biochemical Targets
\section{Biochemical Targets}


% fungal infection
\subsection{Fungal Infection}

Evidence has been put forth that fungal infection is detectable
in brain samples from Alzheimer's disease patients.
\cite{alonso2013fungal}
Recently, fungal infection was directly visualized in the brains
of 11 of 11 AD patients and 0 of 10 controls.
\cite{pisa2015different}
Misfolded amyloid proteins may originate from a transmissible infectious process
(like prion disease).
Some proteins behave as prions in yeast and other fungi.
\cite{soto2006amyloids}
Neuroinflammation is associated with innate immunity,
intended to fight infection at the blood-brain barrier (BBB).
\cite{hauwel2005innate}
Fungal infection could explain the connection between
neuroinflammation, innate immunity and the toxic cell
debris associated with AD.
Amyloid-$\beta$ is toxic to yeast cells
as well as nerve cells.
A yeast model of amyloid toxicity has been proposed
because of yeast's susceptibility to the protein.
\cite{treusch2011functional}
Conversely, amyloids are also believed to be
expressed on fungal cell surfaces.
\cite{gebbink2005amyloids}
The relationship between amyloid, prion and fungi is complicated.
\cite{tessier2009unraveling}

\subsection{Innate Immunity}
The innate immune system, preserved from ancient genes,
is involved in response to pathogenic fungi.
\cite{means2009evolutionarily}
The innate immune system in organisms as diverse as humans and plants
is remarkably similar.
\cite{nurnberger2002innate, nurnberger2004innate}
The signalling cascades involved in innate immune functions
are providing insights into infectious disease, autoimmunity and allergy.
The innate immune response to many infectious agents is associated with
inflammation.
\cite{akira2006pathogen}
Inflammation in AD may be mediated by innate immunity,
according to transgenic animal model.
\cite{fassbender2004lps}

Innate immune response to fungal infection is
contributed to by Toll-like receptors (TLRs).
\cite{bellocchio2004contribution}
Immune response to to some pathogenic fungi is dependent
on TLRs.
\cite{viriyakosol2005innate}

Pathogen associated molecular patterns (PAMPs) are
associated with both fungal infection
\cite{kumar2011pathogen}
and AD.
\cite{salminen2009inflammation}




%Resveratrol is a promising neuroprotective substance.
%\cite{li2012resveratrol}


\subsection{Amyloid-$\beta$}


An animal model found ZMT effective in reducing Amyloid-$\beta$
toxictiy.
\cite{tohda2003repair}


\subsection{Inflammation}


Long-term use of Non-Steroidal Antiinflammatory Drugs (NSAIDs)
have been associated with a dramatic decrease in the risk of AD
in a large cohort.
The risk ratio was 0.20 (95 percent confidence interval, 0.05 to 0.83).
\cite{in2001nonsteroidal}

Other epidemiological evidence suggests that individuals
with diets right in antiinflammatory Curmcumin
are at lower risk of developing AD.
\cite{?}


Inflammation may be the key to initiating the toxic cascade that
characterized Alzheimer's rather than the result of
midfolded proteins such as amyloid and tau.
\cite{heneka2007inflammatory}







\subsection{Cholinergic Drugs}

Cholinergic drugs were able to slow the progression of AD
after 2 years of follow-up.
Although, cognitive deterioration was not stopped.
\cite{requena2006effects}




\subsection{Hormonal Therapy}
Postmenopausal hormone therapy was not found
to affect AD risk in a systematic review of
epidemiological studies.
\cite{o2014postmenopausal}



% Challenges
\section{Challenges}

A major concern raised about crude natural products is
the challenge of standardization.


A formulation called kamikihito (KKT) in Japan
and kami-guibi-tang (KGT) in Korea
were comparable in enhancing memory in an animal model,
although chromatography revealed differences in the formulations.
\cite{watari2015comparing}



% Proposed Formulation


% Conclusion
\section{Conclusion}
Write your conclusion here.










\bibliographystyle{unsrt}
\bibliography{citations}

\end{document}



