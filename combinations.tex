\section{Evidence for Combinations of Natural Products}

Many different traditional mixtures exist which are used
treat AD. The mixtures have many overlapping components.

Table \ref{table:mixtures} shows the degree of overlap between
individual natural products and some traditional mixtures.
This table does not come close to showing the full breadth of
what is used in Chinese medicine.

A review of traditional Chinese formulations
identified 104 formulations used to treat senile dementia
involving 147 kinds of Chinese medicine.
The pair most used was \textit{Ligusticum} and \textit{Acorus},
present in 27.9\% of the identified formulae.
% need to read this!
\cite{zong2014analysis}

In a review of double blind clinical trials of
Chinese medicine to improve cognitive function,
the most commonly used ingredients found were
Acorus,
Panax,
Polygala,
and Poria.

Our review so far has found that
Poria,
Acorus,
Citrus,
Glycyrrhiza and
Zingiberis are the most commonly
found in traditional medicines used in Asia to treat AD.




Saussureae Radix (Saussurea lappa Clarke),
Longanae Arillus (Dimocarpus longana),
Gardeniae Fructus (Gardenia jasminoides Ellis).
Saussurea, Dimocarpus, and Gardenia are unique to KKT.


\begin{table*}[htp]
\centering

\begin{tabular}{||c c c c c c c c c c c c c||}
 \hline
 Genus          & YKS & DTD & FMJ & PN-1 & SZL & WD & BDW & CMT & HCKT & KRBT & CTS & KKT  \\
 \hline\hline
 %              &     &     &     &      &     &    &     &     &      &      &     \\
 % Fungi
 Poria          &  X  &  X  &  X  &      &  X  & X  &  X  &  X  &      &      &  X  &  X   \\
 % Asteraceae
 Atractylodis   &  X  &     &     &      &     &    &     &     &  X   &      &     &  X   \\
 % Apiaceae
 Ligusticum     &  X  &     &     &      &     &    &     &     &      &      &     &     \\
 Angelica       &  X  &     &     &      &     &    &     &     &  X   &      &     &  X   \\
 Bupleurum      &  X  &     &     &      &     &    &     &     &  X   &      &     &  X   \\
 % Fabaceae
 Glycyrrhiza    &  X  &     &     &      &     &    &     &     &  X   &  X   &  X  &  X   \\
 Astragalus     &     &     &     &  X   &     &    &     &     &  X   &      &     &  X   \\
 % Rubiaceae
 Uncariae       &  X  &     &     &      &     &    &     &     &      &      &  X  &     \\
 % Araceae
 Arisaema       &     &  X  &     &      &     &    &     &     &      &      &     &     \\
 Pinellia       &     &  X  &     &      &     & X  &     &  X  &      &      &     &     \\
 % Rutaceae
 Citrus         &     &  X  &  X  &      &     & X  &     &     &  X   &      &  X  &     \\
 % Acoraceae (formerly Araceae)
 Acorus         &     &  X  &  X  &      &  X  & X  &     &  X  &      &      &     &     \\
 % Araliaceae
 Ginseng        &     &  X  &     &      &     &    &     &     &  X   &      &  X  &  X   \\
 % Poaceae
 Bambusa        &     &  X  &     &      &     & X  &     &     &      &      &     &     \\
 % Zingiberaceae
 Zingiberis     &     &  X  &     &      &     &    &     &     &  X   &  X   &  X  &  X   \\
 % Orobanchaceae
 Rehmannia      &     &     &  X  &      &     &    &  X  &     &      &      &     &     \\
 Gastrodia      &     &     &     &      &     & X  &     &     &      &      &     &     \\
 % Asparagaceae
 Ophiopogon     &     &     &  X  &      &     &    &     &     &      &      &  X  &     \\
 Anemarrhena    &     &     &  X  &      &     &    &     &     &      &      &     &     \\
 % Paeoniaceae
 Paeonia        &     &     &  X  &      &     &    &  X  &     &      &  X   &     &     \\
 % Orchidaceae
 Dendrobium     &     &     &  X  &      &     &    &     &     &      &      &     &     \\
 % Lardizabalaceaer
 Akebia         &     &     &  X  &      &     &    &     &     &      &      &     &     \\
 % Orobanchaceae
 Cistanche      &     &     &     &   X  &     &    &     &     &      &      &     &     \\
 % Campanulaceae
 Codonopsis     &     &     &     &      &  X  &    &     &     &      &      &     &     \\
 % Lauraceae
 Cinnamomum     &     &     &     &      &  X  &    &  X  &     &      &   X  &     &     \\
 % Polygalaceae
 Polygala       &     &     &     &      &  X  &    &     &  X  &      &      &     &  X   \\

 % note: SZL and KRBT contain fossil bones, Ossa Draconis
 % Oyster shell ...
 Ostrea         &     &     &     &      &     &    &     &     &      &   X  &     &     \\
 % Polygonaceae
 Polygonum      &     &     &     &      &     & X  &     &     &      &      &     &     \\
 % Rhamnaceae - jujube
 Ziziphus       &     &     &     &      &     & X  &     &     &  X   &   X  &     &  X   \\
% &     &     &     &      &     &    &     &     &      &      &     &     \\
% &     &     &     &      &     &    &     &     &      &      &     &     \\
% &     &     &     &      &     &    &     &     &      &      &     &     \\
% &     &     &     &      &     &    &     &     &      &      &     &     \\

 % Ranunculaceae
 Aconitum       &     &     &     &      &     &    &  X  &     &      &      &     &     \\



 Saussurea      &     &     &     &      &     &    &     &     &      &      &     &  X   \\
 Dimocarpus     &     &     &     &      &     &    &     &     &      &      &     &  X   \\
 Gardenia       &     &     &     &      &     &    &     &     &      &      &     &  X   \\

 \hline
\end{tabular}
\caption{Mixture to Genus Matrix}
\label{table:mixtures}
\end{table*}


% Combinations

% Yokukansan
\subsection{Yokukansan (YKS)}
Yokukansan is a mixture of seven crude natural products,
Atractylodis lanceae rhizoma,
Poria cocos sclerotia,
Cnidii rhizoma,
Angelicae radix,
Bupleuri radix,
Glycyrrhizae radix,
and Uncariae uncis cum ramulus.
Yokukansan is called Yi-Gan San in TCM.
\cite{iwasaki2005randomized}
Clinical trials have demonstrated Yokukan-san’s efficacy
in treating patients with BPSD. Accordingly,
Yokukansan has been listed by The Japanese Society of Neurology
in the Japanese Guidelines for the Management of Dementia
since 2010.
A recent review found 13 clinical trials of varying quality
included a total of 466 patients that found YKS to be a safe
and effective way to raise Neuropsychiatric Inventory (NPI) scores.
The Mini-Mental State Examination (MMSE) score of cognitive impairment
and the Disability Assessment for Dementia (DAD) score of caregiver burden
were unimproved, however.
\cite{okamoto2014yokukan}
Another review of noted improvements in the activities of daily living (ADL) score.
\cite{matsuda2013yokukansan}
Repeated clinical trials have proven the efficacy of YKS in improving
NPI and ADL scores in patients with AD.
\cite{mizukami2014kampo}


In a cross-over study of 106 patients to investigate the use of Yokukansan
to treat the behavioural and psychological symptoms of dementia (BPSD),
a significant improvement in
Neuropsychiatric Inventory (NPI) was found.
Significant improvements were observed in delusions,
hallucinations, agitation/aggression, depression, anxiety, and irritability/lability.
Effects were sustained for one month after treatment was ceased.
Cognitive function was not significantly improved.
\cite{mizukami2009randomized}
In a 52 patient RCT of Yokukansan to treat dementia,
BPSD and ADL scores were improved.
\cite{iwasaki2005randomized}
In an open-label study of 26 patients who received
7.5 grams/day of Yokukansan for 4 weeks,
success was seen in reducing
hallucinations, agitation, anxiety, irritability
and abnormal behavior.
But, overall disability and congnitive function were not improved.
The mixture was well tolerated.
\cite{hayashi2010treatment}
In a non-blinded, randomized, parallel-group comparison study
with 61 participants, YKS with donepezil was better than donepezil alone
in measures BPSD. Proving YKS to be safe and effective.
\cite{okahara2010effects}


Yokukansan is considered very safe.
In a case series of 3 patients between 10 and 13 years old,
YKS considered effective in treating pediatric emotional and behavioral problems
within 14-21 days.
\cite{tanaka2013potential}


\subsection{Di-Tan Decoction (DTD)}
DTD is a combination of ...

A double-blind, randomized, placebo-controlled, add-on trial
testing the efficacy of DTD to treat cognitive impairment
in AD patients.
\cite{chua2015efficacy}


\subsection{Shen-Zhi-Ling (SZL)}
SZL is an oral liquid consisting of 10 kinds of traditional Chinese medicine:
Codonopsis pilosula, Cassia Twig, Paeonia lactiflora,
honey-fried Licorice root, Poria Cocos, Rhizoma Zingiberis, Radix Polygalae,
Acorus tatarinowii, Ossa Draconis, and Concha Ostreae.

98 patients completed a double-blind clinical trail of SZL.
SZL was found to be more effective than placebo,
delaying BPSD and improving scores of evening activity
and nocturnal activity.
\cite{pan2014shen}




\subsection{Keishi-ka-ryukotsu-borei-to (KRBT)}

KRBT is a mixture of 7 natural products:
cinnamon bark,
peony root,
jujube fruit,
oyster shell,
fossilized bone,
glycyrrhiza,
and ginger rhizome
that was found to effectively BPSD in a case report.
Gonadotrophin profiles were positively altered.
\cite{niitsu2013behavioural}




\subsection{Chotosan (CTS)}
Chotosan is a mixture of 11 natural products,
Uncariae Uncis cum Ramulus,
Aurantii Nobilis pericarpium,
Pinelliae tuber,
Ophiopogonis tuber,
Poria cocos,
Ginseng radix,
Saposhnikoviae radix,
Chrysanthemi flos,
Glycyrrhizae radix,
Zingiberis rhizome,
and Gypsum fibrosum.
Along with YKS, Chotosan has shown promise in clinical and
preclinical studies as a treatment for AD.
Both mixtures contain Uncariae Uncis.
\cite{matsumoto2013kampo}



\subsection{Kamikihito (KKT)}
KKT is composed of 14 crude drugs:

Ginseng Radix (P. ginseng C.A. Meyer),
Polygalae Radix (P. tenuifolia Willd.),
Astragali Radix (A. membranaceus Bunge),
Zizyphi Fructus (Zizyphus jujube Mill. var. inermis Rehd.),
Zizyphi Spinosi Semen (Z. jujube Mill. var. spinosa),
Angelicae Radix (Angelica acutiloba Kitagawa),
Glycyrrhizae Radix (Glycyrrhiza uralensis Fisch),
Atractylodis Rhizoma (Atractylodes japonica Koidzumi ex Kitamura),
Zingiberis Rhizoma (Zingiber officinale Roscoe),
Poria (Poria cocos Wolf),
Saussureae Radix (Saussurea lappa Clarke),
Longanae Arillus (Dimocarpus longana),
Bupleuri Radix (Bupleurum falcatum Linne), and
Gardeniae Fructus (Gardenia jasminoides Ellis).
Eleven of the fourteen NPs are listed in \ref{table:mixtures}.
Saussurea, Dimocarpus, and Gardenia are unique to KKT.

In a small, unblinded clinical trial, KKT improved cognitive impairment in
patients with mild dementia.
\cite{arai2015effectiveness}

An animal model revealed that KKT
improved amyloid-$\beta$-induced tau phosphorylation and axonal atrophy
even after axonal degeneration had progressed.
\cite{watari2014new, watari2015comparing}



\subsection{Ninjin'yoeito (NYT)}


23 patients who had a insufficient response to donepezil
received donepezil alone or donepezil and NYT.
A 2-year follow-up showed that patients receiving NYT
had an improved cognitive outcome and alleviation of AD-related depression.
\cite{kudoh2015effect}





\subsection{Hochuekkito}
Hochuekkito is a mix of 10 natural products,
Astragali Radix,
Atractylodis lanceae Rhizoma,
Ginseng Radix,
Angelicae Radix,
Bupleuri Radix,
Zizyphi Fructus,
Aurantii Bobilis Pericarpium,
Glycyrrhizae Radix,
Cimicifugae Rhizoma,
and Zingiberis Rhizoma.
\cite{kiyohara2011polysaccharide}

In a placebo-controlled clinical trial of Hochuekkito
showed that the formulation improved the QOL and immunological status
of elderly patients with weakness.
\cite{satoh2005randomized}

\subsection{Zokumei-to (ZMT)}

In an animal model of AD using Amyloid-$\beta$,
ZMT treatment significantly increased the level of expression of
synaptophysin up to the control level.
Memory impairment and synaptic loss was ameliorated in the mice
even after impairment had progressed.
\cite{tohda2003repair}



\subsection{Ba Wei Di Huang Wan (BWD)}
BWD is a traditional Chinese formulation of 8 natural products,
Rehmannia glutinosa,
Cornus officinalis,
Dioscoreabatatas root,
Alisma orientale rhizome,
Poria cocos,
Paeonia suffruti-cosa,
Cinnamomum cassia,
and Aconitum carmichaeli.

In a placebo-controlled RCT of 33 patients with AD,
cognitive function was significantly improved by BWD compared to placebo
based on the Mini-Mental State Examination (MMSE).
The activities of daily living (ADLs) score was also improved
versus placebo.
Scores returned to baseline after eight weeks.
\cite{iwasaki2004randomized}




\subsection{Kai-xin-san (KXS)}
KXS is a traditional formulation thought to be beneficial in the treatment
of AD.
It contains
Panax,
Polygala,
Acorus,
and Poria in at least 3 published ratios.
First described around 650 by Sun Simiao,
it is one of the most popular mixtures for depression in TCM
and increased neurotrophic factors in cultured astrocytes.
\cite{zhu2013kai}
KXS modulated neurological parameters in an animal model of depression.
\cite{zhu2012standardized}
The Chinese report progress in using KXS against AD.
\cite{wen2013research}
KXS was thought to improve learning and memory in an animal model of dementia.
\cite{li2009effects}






\subsection{Yishen Huazhuo decoction (YHD)}
YHD was found to be comparable or better than
the conventional AChE drug donepezil

\cite{zhang2015cognitive}


\subsection{And more?}


Xixin Decoction is a mixture of
Ginseng Radix et Rhizoma,
Pinelliae Rhizoma,
Poria,
Aconiti lateralis Radix praeparata,
et al(?)
that was effective in an animal model of AD.
\cite{diwu2013effect}



Juzen-taiho-to (JTT), also known as
Shi quan da bu tang in Chinese medicine
is a mixture of 10 familiar ingredients,
Panax ginseng (Ginseng),
Angelica sinensis (Dong quai),
Paeonia lactiflora (Peony),
Atractylodes macrocephala (Atractylodes),
Poria cocos (Hoelen),
Cinnamomum cassia (Cinnamon),
Astragalus membranaceus (Astragulus),
Liqusticum wallichii (Cnidium),
Glycyrrhiza uralensis (Licorice),
and Rehmannia glutinosa (Rehmannia).
JTT has been investigated in an animal model
for its potential to help against \textit{Candida}
infection.
\cite{akagawa1996protection}
In another model that tested components individually,
Ginseng, Glycyrrhizae radix, Atractylodis and Cnidii
were found to be the promising antifungal components individually.
\cite{abe1998protective}



An herbal formula consisting
of aqueous extracts of
\textit{Poria cocos}, \textit{Atractylodes macrocephala}
and \textit{Angelica sinensis} worked on
an animal model of AD, reducing AChE activity.
\textit{Angelica} was the most cholinergic of the 3 ingredients.
\cite{lin2009aqueous}


