\begin{abstract}
%(Motivation)
Alzheimer's disease (AD) is a chronic neurodegenerative condition
associated with old age. It is the most common cause of dementia,
inflicting an increasing social and economic burden.
Medications which are currently approved in the United States
do not reverse or prevent disease progression.
Therefore, we seek new therapies.

%(Natural Products)
There has been abundant interest in natural products to treat Alzheimer's.
A large number of preclinical studies have demonstrated
that crude natural products have therapeutic potential.
%But, in some cases, preclinical results have failed to
%translate into significant findings in randomized,
%placebo-controlled clinical trials (RCTs).
%(Traditional Medicine)
Clinical trials have been conducted on a few of these individual natural products.
A number of traditional medicines which are combinations of natural products
have been used to treat AD.
In modern times, testing has been undertaken to scientifically verify the
activities of these combinations, including some successful clinical trials.
Ascertaining which constituents are responsible for the positive effects,
what interactions may be involved,
and what role synergism plays, are daunting combinatorial problems.

%(AD Causation)
Prevailing theories of AD causation have focused on the observed
biochemical abnormalities of the senile brain.
Observations of amyloid-$\beta$ plaques, cholinergic deficits,
tau protein abnormalities,
neuroinflammation and oxidative stress,
and aberrant signalling cascades leading to neuronal apoptosis
have partially elucidated the clinical picture
but have failed to produce satisfactory pharmaceuticals.
Natural products are rich in substances which target
relevant pathways in AD.
% Galantamine
Some have even become FDA-approved treatments for AD.

Recent research has produced evidence
that AD may be caused by fungal infection in the central nervous system (CNS).
Many natural products which have been investigated for their potential
to curb the biochemical abnormalities associated with AD
also show promise as antifungal agents.

%(Summary)
We review the available literature in an attempt
to determine which natural products are the most promising
candidates for AD therapy.
Preferred candidates have success in clinical trial,
wide safety margins and robust biochemical rationale.
We propose a combination to be tested in a
randomized, placebo-controlled clinical trial (RCT).
\end{abstract}

