%\input{etiology}
\section{Background}

% Natural medicines are important and viable
Natural products have probably been used as medicine for over 60,000 years.
From then until modern times,
more than 200,000 natural products and compounds have been discovered.
\cite{ji2009natural}
In recent history, modern chemistry has allowed the chemical design of
individual molecules for pharmaceutical applications.
But, applying individual molecules to chronic neurodegenerative diseases, for instance,
has brought scant success.
Interest in natural medicine has sustained through the modern era.
A large majority of drugs derive from natural products.
There were no
%de novo combinatorial
fully synthetic compounds
approved as a drugs in the time frame
between 1981 and 2002.
\cite{newman2003natural}
Of 175 small molecules used against cancer up to 2010,
74.8\% were not completely synthetic and
48.6\%, were directly derived from natural products.
\cite{newman2012natural}
Antibiotic and antifungal compounds have come from leads provided by nature.
Antibiotic discovery has been dependent on metabolites produced by soil bacteria.
% Willow bark, the source of aspirin, was used from ancient times.
\cite{laursen2004phenazine}
Ethnobotany still provides a rich resevoir of CNS-active pharmacological leads.
\cite{mcclatchey2009ethnobotany, perry1999medicinal}
Herbal compounds from traditional medicine represent a frontier
in dementia pharmacology research.
\cite{jesky2011herbal}

% Combinations and synergism
Traditional medicines from China, Japan and Korea invariably employ combinations
of different materials from plants and fungi.
A large number of active components have been identified.
\cite{gao2013research}
Synergistic relationships have been shown between substances
in traditional combinations.
Multiple active components working together may be the key
to future AD treatments.
\cite{kong2009hope, liu2014history}

% Alzheimer's in particular
Psycotropic medicines derived from natural products are particularly promising.
\cite{lake2000psychotropic}
Significant evidence exists that suggests that NPs
may be effective psychotherapeutics.
\cite{fugh1999dietary}
A wealth of NPs are candidates for the treatment of Alzheimer's in particular.
\cite{houghton2005natural}
% Specific molecules, targets
Many acetylcholinesterase(AChE) inhibitors with potential clinical relevance
have been discovered from natural products.
\cite{barbosa2006natural}
NPs may play a role in inhibiting microglial neurotoxicity.
\cite{choi2011inhibitors}
In addition to biochemical target activities,
traditional medicines may provide improvements in
cognitive impairment, fatigue, mood, and anxiety.
\cite{divino2011role}
% class 1 evidence overview
A review of RCTs of ``herbal medicine" for dementia in 2009 found
13 trials and concluded that some herbal medicines were more effective than
placebo and at least as effective as standard drugs.
\cite{may2009herbal}

Some have recently concluded,
through a review of the scientific literature,
that western pure drugs
can not replace the advantages of Chinese combinations
in AD treatment.
\cite{su2014treatment}



%%%%%%%%%%%%%%%%%%%%%%%%%%%%%%%%%
% AD Etiology
%%%%%%%%%%%%%%%%%%%%%%%%%%%%%%%%%


\subsection{Alzheimer's Disease Etiology}

AD is a disease of old age
characterized by slow onset with a
progressive loss of cognitive function and gradual decline into dementia.
AD is the most common cause of dementia.
The time from diagnosis until death is usually 8-10 years.
The most notable pathological changes are plaques of misfolded proteins,
neurofibrillary tangles, alterations in the microvasculature of the brain,
inflammation and oxidative stress.

\subsubsection{Amyloid-$\beta$ (AB)}

The cerebral deposition of amyloid-$\beta$ protein in AD patients
has been recognized for decades.
These proteins, which begin to accumulate
for years or even decades before the onset of dementia,
 have long been considered central to the pathology of AD.
\cite{citron1992mutation}

A popular hypothesis states that AB is the causative agent in AD,
causing the cascade of abnormality observed. Genetic mutations
which cause abnormal amyloid precursor protein (APP) are implicated.
AB is neurotoxic and could lead to the neurofibrillary tangles
and ultimately the death of neurons.
\cite{hardy1992alzheimer}


Soluble AB was strongly correlated with AD severity.
Measures of insoluble AB could distinguish AD patients from
controls but could not predict AD severity.
AB is usually thought of as extracellular.
However, soluble AB can be extracellular or intracellular.
\cite{mclean1999soluble}


More recently, synaptotoxic AB oligomers have been implicated
in AD biopathology,
triggering the accumulation of reactive oxygen species (ROSs).
This provides an explanation for the therapeutic action
of memantine.
\cite{klein2013synaptotoxic}


% AChE
\subsubsection{Cholinergic}

Long ago scientists observed deficits in cholinergic innervation
and loss of cholinergic neurons in AD patients.
\cite{davies1976selective,
coyle1983alzheimer}





% ApoE
% Tau
% genetics


\subsubsection{Microcirculation}

An idea related to AB toxicity is that disturbed microcirculation in the brain
causes AD.
\cite{de1993can}
One recent review stated that the AB hypothesis of AD causation has
proven inadequate and that therapeutic strategies should
focus on the microcirculation hypothesis and supporting normal angiogenesis.
\cite{drachman2014amyloid}

The microcirculation hypthesis states that the primary cause of
AD is related to deficiencies in the microvasculature of the brain.
These deficits may allow the inflow of neurotoxins into the brain
and, perhaps more importantly, prevent the adequate outflows of
neurotoxins such as AB.

Blood-brain barrier (BBB) dysfunction has been noted in AD patients,
including leakage which may allow
neurotoxins and infectious agents to enter the brain.
However, this finding is controversial.
Some studies failed to find additional leakage in AD brains
when compared to age-matched controls.
Other BBB dysfunctions observed in AD involve inadequate nutrient supply and
inadequate clearance of toxic substances.
In addition, altered proteins at the neuro-vascular unit may promote
inflammation, oxidative stress and neuronal damage.
\cite{erickson2013blood}


\subsubsection{Inflammation}

Certainly AD pathology involves extracellular plaques involving
AB, neurofibrillary tangles with abnormal tau protiens,
vascular malfunction
and cell death caused by ROS and inflammation.
Some propose that inflammation and oxidative stress precede
the development of the AB plaques, initiating the
pathological cascade observed in AD.
\cite{luque2014oxidative}




\subsubsection{Infection}

The idea that infections cause AD has been widely studied and
has not been ruled out. In fact,
a substantial amount of evidence suggests that
AD may be caused by infection.
Histopathological hallmarks of AD are known to occur in
chronic infections, including those of the CNS.
\cite{
iked1995numerous,
mandybur1990distribution,
kueh1984amyloid,
liberski1994transmissible,
kobayashi2008plaque,
sikorska2009ultrastructural,
de1984serum,
looi1988immunohistochemical,
rocken1999generalized,
wangel1982family,
tank2000renal,
urban1993ct}
In addition, evidence has emerged that AB is antimicrobial
and may be produced as an innate immune response to infection.
% sweet paper.
\cite{soscia2010alzheimer}


Inflammatory cytokines typical of immune response to infection
are associated with the inflammation observed in AD.
Elevated levels of interleukin-1, interleukin-6 and
tumor necrosis factor-$\alpha$ have been observed.
\cite{
sastre2006contribution,
holmes2011systemic,
akiyama2000inflammation,
cojocaru2011study}

Pathogens most commonly studied in AD etiology have been
viral, bacterial and prional. These include
HSV-1,
\textit{C. pneumoniae},
\textit{B. burgdorferi},
\textit{H. pylori},
and prions.
One group has argues that HSV-1 initiates the pathological
cascade.
\cite{ball2013intracerebral}
Although, if HSV-1 is the primary cause of AD,
how do some people without HSV-1 infection develop AD?
Another group found that anti-HSV IgM was associated with
double the risk of developing AD, indicating that
reactivated HSV-1 infection increases the risk of developing AD.
\cite{lovheim2014reactivated}

The possibility of fungal pathogenisis
has been almost completely overlooked.
\cite{mawanda2013can}

Recently, a series of studies provided evidence
that AD may be caused by fungal infection in the CNS
\cite{
pisa2015different,
alonso2013fungal,
pisa2015direct}




\subsubsection{Other theories of causation}

Deficits in glucose metabolism including
insulin resistance have been implicated as
risk factors for AD.
\cite{erickson2013blood}
Some have even proposed that
these irregularities could play a role in
the etiology of the disease.
\cite{luque2014oxidative}
Research has indicated that those who develop
type 2 diabetes and those who develop AD
share some genetic risk factors.
\cite{hao2015shared}


One review found that AD may be related to sleep apnea.
\cite{pan2014can}


% full subsection for genetics?
In addition to a number of possible environmental hazards,
an epidemiological review found a number of genes which
increased the risk of early-onset and late-onset AD.
\cite{jiang2013epidemiology}


%\input{traditions}
%%%%%%%%%%%%%%%%%%%%%%%%%%%%%%%%%%%%%%%%%%%%%%%%%%

\subsection{Traditions}
% Kampo
\subsubsection{Kampo}

Kampo medicine, a traditional medicine of Japan,
is a complex system of individualized diagnosis and treatment.
An important component of Kampo is precise formulations of natural products
based on older Chinese recipes.
148 or more combinations are covered by the national health insurance system in Japan.
Of 135 published randomized controlled trials on Kampo combinations
published between 1986 and 2007,
37 were available in English.
As chronic degenerative diseases have become more prominent in an aging population,
interest in Kampo has increased as has Kampo's integration into modern medical practice.
A recent survey indicated that 70\% of Japanese physicians prescribe Kampo formulations.
% About 2% of prescriptions are Kampo /citation? ~ 200,000 a year
Insurance-covered formulations include standardized extracts and crude decoctions.
Applying Kampo formulations to conventional diagnoses is a challenge because
in traditional Kampo, the same conventional diagnosis could result in the prescription
of different formulations while different conventional diagnoses could result
in the prescription of the same formulation.
\cite{watanabe2011traditional}
However, Kampo may evolve based on new scientific evidence.
\cite{terasawa2004evidence}

The testing specfic individual formulations on a specific conventional diagnosis
is useful for applying Kampo formulations in clinical practice in
countries like the United States where expert Kampo practioners are few.
Therefore, for the purposes of this review, the best evidence will be considered
to be randomized, placebo-controlled clinical trials on specific formulations
for the specific diagnosis of AD.

\subsubsection{TCM}

A review of Chinese Herbal Medicine for the management of vascular dementia (VD),
a disease similar to AD,
was conducted, highlighting the urgent need for more clinical trials of high quality.
% long, detailed PhD thesis from Australia
\cite{liu2008development}

\subsubsection{Mixtures Used in Other Countries}


Korean traditional medicines are heavily influenced
by Chinese and Japanese. Accordingly,
the most commonly used mixtures used to treat dementia
in Korea are formed from the same ingredients found in
Kampo and TCM.
\cite{jo2014tendency}

Ayurvedic medicine, the traditional medicine of India,
has had detailed written descriptions of Alzheimer's dementia
for millenia.
\cite{manyam1999dementia}
Some of the ayurvedic pharmacopeia coincides with those
of East Asia.


